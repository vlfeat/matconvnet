\documentclass[12pt]{memoir}
\usepackage{amssymb}
\usepackage{amsmath}
\usepackage{url}
\usepackage{xspace}
\usepackage[margin=2.5cm]{geometry}
\usepackage{tikz}
\usepackage{pgfplots} 
\usepackage{listings}
\usepackage{color}
\usepackage{textcomp}
\usepackage{fancyvrb}
\usepackage[colorlinks]{hyperref}

\newcommand{\real}{\mathbb{R}}
\newcommand{\vv}{\operatorname{vec}}
\newcommand{\diag}{\operatorname{diag}}
\newcommand{\vlnn}{\textsc{MatConvNet}\xspace}
\newcommand{\cpp}{C{}\texttt{++}~}
\newcommand{\bx}{\mathbf{x}}
\newcommand{\by}{\mathbf{y}}
\newcommand{\bz}{\mathbf{z}}
\newcommand{\bff}{\mathbf{f}}
\newcommand{\br}{\mathbf{r}}
\newcommand{\bw}{\mathbf{w}}
\newcommand{\bfs}{\mathbf{s}}
\newcommand{\bfe}{\mathbf{e}}
\newcommand{\bone}{\mathbf{1}}
\newcommand{\argmin}{\operatornamewithlimits{argmin}}
\newcommand{\argmax}{\operatornamewithlimits{argmax}}
\newcommand{\sign}{\operatornamewithlimits{sign}}

\tikzstyle{block} = [draw, rectangle, minimum height=3em, minimum width=3em] \tikzstyle{data} = []
\tikzstyle{pinstyle} = [pin edge={to-,thin,black}]

\VerbatimFootnotes

\setsecnumdepth{subsection}
\settocdepth{subsection}

\definecolor{listinggray}{gray}{0.9}
\definecolor{lbcolor}{rgb}{0.8,0.8,0.8}
\lstset{
	%backgroundcolor=\color{lbcolor},
	tabsize=4,
	rulecolor=,
	language=matlab,
	basicstyle=\footnotesize,
	upquote=true,
	columns=fixed,
	aboveskip={1.2\baselineskip},
	belowskip={1.2\baselineskip},
	showstringspaces=false,
	extendedchars=true,
	breaklines=false,
	prebreak = \raisebox{0ex}[0ex][0ex]{\ensuremath{\hookleftarrow}},
	%frame=single,
	showtabs=false,
	showspaces=false,
	showstringspaces=false,
	identifierstyle=\ttfamily,
	keywordstyle=\color[rgb]{0,0,1},
	commentstyle=\itshape\color[rgb]{0.133,0.545,0.133},
	stringstyle=\color[rgb]{0.627,0.126,0.941},
}

\title{MatConvNet \\
\Large
Convolutional Neural Networks for MATLAB}
\author{
Andrea Vedaldi
\and
Karel Lenc}
\date{}

% ------------------------------------------------------------------
\begin{document}
% ------------------------------------------------------------------

\frontmatter
\maketitle{}
\clearpage

\begin{abstract}
\vlnn is an implementation of Convolutional Neural Networks (CNNs) for MATLAB. The toolbox is designed with an emphasis on simplicity and flexibility. It exposes the building blocks of CNNs as easy-to-use MATLAB functions, providing routines for computing linear convolutions with filter banks, feature pooling, and many more. In this manner, \vlnn allows fast prototyping of new CNN architectures; at the same time, it supports efficient computation on CPU and GPU allowing to train complex models on large datasets such as ImageNet ILSVRC. This document provides an overview of CNNs and how they are implemented in \vlnn and gives the technical details of each computational block in the toolbox.
\end{abstract}
\clearpage

\tableofcontents*
\clearpage

\mainmatter
% ------------------------------------------------------------------
\chapter{Introduction to MatConvNet}\label{s:intro}
% ------------------------------------------------------------------

\matconvnet is a MATLAB toolbox implementing \emph{Convolutional Neural Networks} (CNN) for computer vision applications.  Since the breakthrough work of~\cite{krizhevsky12imagenet}, CNNs have had a major impact in computer vision, and image understanding in particular, essentially replacing traditional image representations such as the ones implemented in our own VLFeat~\cite{vedaldi10vlfeat} open source library.

While most CNNs are  obtained by composing simple linear and non-linear filtering operations such as convolution and rectification, their implementation is far from trivial. The reason is that CNNs need to be learned from vast amounts of data, often millions of images, requiring very efficient implementations. As most CNN libraries, \matconvnet achieves this by using a variety of optimizations and, chiefly, by supporting computations on GPUs.

Numerous other machine learning, deep learning, and CNN open source libraries exist. To cite some of the most popular ones: CudaConvNet,\footnote{\small\url{https://code.google.com/p/cuda-convnet/ }} Torch,\footnote{\small\url{http://cilvr.nyu.edu/doku.php?id=code:start}} Theano,\footnote{\small\url{http://deeplearning.net/software/theano/}} and Caffe\footnote{\small\url{http://caffe.berkeleyvision.org}}.  Many of these libraries are  well supported, with dozens of active contributors and large user bases. Therefore, why creating yet another library?

The key motivation for developing \matconvnet was to provide an environment particularly friendly and efficient for researchers to use in their investigations.\footnote{While from a user perspective \matconvnet currently relies on MATLAB, the library is being developed with a clean separation between MATLAB code and the C++ and CUDA core; therefore, in the future the library may be extended to allow processing convolutional networks independently of MATLAB.} \matconvnet achieves this by its deep integration in the MATLAB environment, which is one of the most popular development environments in computer vision research as well as in many other areas. In particular, \matconvnet exposes as simple MATLAB commands CNN building blocks such as convolution, normalisation and pooling (\cref{s:blocks}); these can then be combined and extended with ease to create CNN architectures. While many of such blocks use optimised CPU and GPU implementations written in C++ and CUDA (section~\cref{s:speed}), MATLAB native support for GPU computation means that it is often possible to write new blocks in MATLAB directly while maintaining computational efficiency. Compared to writing new CNN components using lower level languages, this is an important simplification that can significantly accelerate testing new ideas. Using MATLAB also provides a bridge towards other areas; for instance, \matconvnet was recently used by the University of Arizona in planetary science, as summarised in this NVIDIA blogpost.\footnote{\small\url{http://devblogs.nvidia.com/parallelforall/deep-learning-image-understanding-planetary-science/}}

\matconvnet can learn large CNN models such AlexNet~\cite{krizhevsky12imagenet} and the very deep networks of~\cite{simonyan14deep} from millions of images. Pre-trained versions of several of these powerful models can be downloaded from  the \matconvnet home page\footnote{\small\url{http://www.vlfeat.org/matconvnet/}}. While powerful, \matconvnet remains simple to use and install. The implementation is fully self-contained, requiring only MATLAB and a compatible C++ compiler (using the GPU code requires the freely-available CUDA DevKit and a suitable NVIDIA GPU). As demonstrated in \cref{f:demo} and \cref{s:getting-statrted}, it is possible to download, compile, and install \matconvnet using three MATLAB commands. Several fully-functional examples demonstrating how small and large networks can be learned are included. Importantly, several \emph{standard pre-trained network} can be immediately downloaded and used in applications. A manual with a complete technical description of the toolbox is maintained along with the toolbox.\footnote{\small\url{http://www.vlfeat.org/matconvnet/matconvnet-manual.pdf}} These features make \matconvnet useful in an educational context too.\footnote{An example laboratory experience based on \matconvnet can be downloaded from {\small\url{http://www.robots.ox.ac.uk/~vgg/practicals/cnn/index.html}}.}

\matconvnet is open-source released under a BSD-like license. It can be downloaded from \url{http://www.vlfeat.org/matconvnet} as well as from GitHub.\footnote{\small\url{http://www.github.com/matconvnet}}.

% ------------------------------------------------------------------
\section{Getting started}\label{s:getting-statrted}
% ------------------------------------------------------------------

\begin{figure}
%\centering
%\includegraphics[width=0.5\columnwidth]{figures/pepper}
\hrule
\begin{lstlisting}[escapechar=!]
% install and compile MatConvNet (run once)
untar(['http://www.vlfeat.org/matconvnet/download/' ...
   'matconvnet-1.0-beta25.tar.gz']) ;
cd matconvnet-1.0-beta25
run matlab/vl_compilenn

% download a pre-trained CNN from the web (run once)
urlwrite(...
 'http://www.vlfeat.org/matconvnet/models/imagenet-vgg-f.mat', ...
 'imagenet-vgg-f.mat') ;

% setup MatConvNet
run matlab/vl_setupnn

% load the pre-trained CNN
net = load('imagenet-vgg-f.mat') ;

% load and preprocess an image
im = imread('peppers.png') ;
im_ = imresize(single(im), net.meta.normalization.imageSize(1:2)) ;
im_ = im_ - net.meta.normalization.averageImage ;

% run the CNN
res = vl_simplenn(net, im_) ;

% show the classification result
scores = squeeze(gather(res(end).x)) ;
[bestScore, best] = max(scores) ;
figure(1) ; clf ; imagesc(im) ;!
\begin{tikzpicture}[overlay]
\node (x) {};
\node (y) at (7,1) {\includegraphics[width=5cm]{figures/pepper}};
\draw [->,thick] (.1,.1) -- (4.5,.1) {};
\end{tikzpicture}!
title(sprintf('%s (%d), score %.3f',...
net.classes.description{best}, best, bestScore)) ;
\end{lstlisting}
\hrule
\caption{A complete example including download, installing, compiling and running  \matconvnet to classify one of  \matlab stock images using a large CNN pre-trained on ImageNet.}
\label{f:demo}
\end{figure}

\matconvnet is simple to install and use. \cref{f:demo} provides a complete example that classifies an image using a latest-generation deep convolutional neural network. The example includes downloading MatConvNet, compiling the package, downloading a pre-trained CNN model, and evaluating the latter on one of \matlab's stock images.

The key command in this example is !vl_simplenn!, a wrapper that takes as input the CNN !net! and the pre-processed image !im_! and produces as output a structure !res! of results. This particular wrapper can be used to model networks that have a simple structure, namely a \emph{chain} of operations. Examining the code of !vl_simplenn! (!edit vl_simplenn! in \matconvnet) we note that the wrapper transforms the data sequentially, applying a number of \matlab functions as specified by the network configuration. These function, discussed in detail in \cref{s:blocks}, are called ``building blocks'' and constitute the backbone of \matconvnet.

%Since blocks are made available as \matlab functions, it is easy for a user to write different wrappers that implement more complex network structures, such as recurrent neural networks. Likewise, it is easy to extend !vl_simplenn! with new building blocks.

While most blocks implement simple operations, what makes them non trivial is their efficiency (\cref{s:speed}) as well as support for backpropagation (\cref{s:back}) to allow learning CNNs. Next, we demonstrate how to use one of such building blocks directly. For the sake of the example, consider convolving an image with a bank of linear filters. Start by reading an image in \matlab, say using !im = single(imread('peppers.png'))!, obtaining a $H \times W \times D$ array !im!, where $D=3$ is the number of colour channels in the image. Then create a bank of $K=16$ random filters of size $3 \times 3$ using !f = randn(3,3,3,16,'single')!. Finally, convolve the image with the filters by using the command !y = vl_nnconv(x,f,[])!. This results in an array !y! with $K$ channels, one for each of the $K$ filters in the bank.

While users are encouraged to make use of the blocks directly to create new architectures, \matlab provides wrappers such as !vl_simplenn! for standard CNN architectures such as AlexNet~\cite{krizhevsky12imagenet} or Network-in-Network~\cite{lin13network}. Furthermore, the library provides numerous examples (in the !examples/! subdirectory), including code to learn a variety of models on the MNIST, CIFAR, and ImageNet datasets. All these examples use the !examples/cnn_train! training code, which is an implementation of stochastic gradient descent (\cref{s:wrappers-learning}). While this training code is perfectly serviceable and quite flexible, it remains in the !examples/! subdirectory as it is somewhat problem-specific. Users are welcome to implement their optimisers.

% ------------------------------------------------------------------
\section{\matconvnet at a glance}\label{s:vlnn}
% ------------------------------------------------------------------

\matconvnet has a simple design philosophy. Rather than wrapping CNNs around complex layers of software, it exposes simple functions to compute CNN building blocks, such as linear convolution and ReLU operators, directly as MATLAB commands. These building blocks are easy to combine into complete CNNs and can be used to implement sophisticated learning algorithms. While several real-world examples of small and large CNN architectures and training routines are provided, it is always possible to go back to the basics and build your own, using the efficiency of MATLAB in prototyping. Often no C coding is required at all to try new architectures. As such, \matconvnet is an ideal playground for research in computer vision and CNNs.

\matconvnet contains the following elements:
\begin{itemize}
\item \emph{CNN computational blocks.} A set of optimized routines computing fundamental building blocks of a CNN. For example, a convolution block is implemented by \linebreak !y=vl_nnconv(x,f,b)! where !x! is an image, !f! a filter bank, and !b! a vector of biases (\cref{s:convolution}). The derivatives are computed as
![dzdx,dzdf,dzdb] = vl_nnconv(x,f,b,dzdy)! where !dzdy! is the derivative of the CNN output w.r.t !y!~(\cref{s:convolution}). \cref{s:blocks} describes all the blocks in detail.
\item \emph{CNN wrappers.} \matconvnet provides a simple wrapper, suitably invoked by !vl_simplenn!, that implements a CNN with a linear topology (a chain of blocks). It also provides a much more flexible wrapper supporting networks with arbitrary topologies, encapsulated in the !dagnn.DagNN! MATLAB class.
\item \emph{Example applications.} \matconvnet provides several examples of learning CNNs with stochastic gradient descent and CPU or GPU, on MNIST, CIFAR10, and ImageNet data.
\item \emph{Pre-trained models.} \matconvnet provides several state-of-the-art pre-trained CNN models that can be used off-the-shelf, either to classify images or to produce image encodings in the spirit of Caffe or DeCAF.
\end{itemize}

% ------------------------------------------------------------------
\section{Documentation and examples}\label{s:examples}
% ------------------------------------------------------------------

\begin{figure}
\centering
\includegraphics[width=0.65\columnwidth]{figures/imnet}
\vspace{-1em}
\caption{Training AlexNet on ImageNet ILSVRC: dropout vs batch normalisation.}\label{f:imnet}
\end{figure}

There are three main sources of information about \matconvnet. First, the website contains descriptions of all the functions and several examples and tutorials.\footnote{\small See also \url{http://www.robots.ox.ac.uk/~vgg/practicals/cnn/index.html}.} Second, there is a PDF manual containing a great deal of technical details about the toolbox, including detailed mathematical descriptions of the building blocks. Third, \matconvnet ships with several examples (\cref{s:getting-statrted}).

Most examples are fully self-contained. For example, in order to run the MNIST example, it suffices to point MATLAB to the \matconvnet root directory and type !addpath examples! followed by !cnn_mnist!. Due to the problem size, the ImageNet ILSVRC example requires some more preparation, including downloading and preprocessing the images (using the bundled script !utils/preprocess-imagenet.sh!). Several advanced examples are included as well. For example, \cref{f:imnet} illustrates the top-1 and top-5 validation errors as a model similar to AlexNet~\cite{krizhevsky12imagenet} is trained using either standard dropout regularisation or the recent \emph{batch normalisation} technique of~\cite{ioffe15batch}. The latter is shown to converge in about one third of the epochs (passes through the training data) required by the former.

The \matconvnet website contains also numerous \emph{pre-trained} models, i.e. large CNNs trained on ImageNet ILSVRC that can be downloaded and used as a starting point for many other problems~\cite{chatfield14return}. These include: AlexNet~\cite{krizhevsky12imagenet}, VGG-S, VGG-M,  VGG-S~\cite{chatfield14return}, and  VGG-VD-16, and VGG-VD-19~\cite{simonyan15very}.  The example code of \cref{f:demo} shows how one such model can be used in a few lines of MATLAB code.

% ------------------------------------------------------------------
\section{Speed}\label{s:speed}
% ------------------------------------------------------------------

Efficiency is very important for working with CNNs. \matconvnet supports  using NVIDIA GPUs as it includes CUDA implementations of all algorithms (or relies on MATLAB CUDA support).

 To use the GPU (provided that suitable hardware is available and the toolbox has been compiled with GPU support), one simply converts the arguments to !gpuArrays! in MATLAB, as in !y = vl_nnconv(gpuArray(x), gpuArray(w), [])!. In this manner, switching between CPU and GPU is fully transparent. Note that \matconvnet can also make use of the NVIDIA CuDNN library with significant speed and space benefits.

Next we evaluate the performance of \matconvnet when training large architectures on the ImageNet ILSVRC 2012 challenge data~\cite{deng09imagenet}. The test machine is a Dell server with two Intel Xeon CPU E5-2667 v2 clocked at 3.30 GHz (each CPU has eight cores), 256 GB of RAM, and four NVIDIA Titan Black GPUs (only one of which is used unless otherwise noted). Experiments use \matconvnet beta12, CuDNN v2, and MATLAB R2015a. The data is preprocessed to avoid rescaling images on the fly in MATLAB and stored in a RAM disk for faster access. The code uses the !vl_imreadjpeg! command to read large batches of JPEG images from disk in a number of separate threads. The driver !examples/cnn_imagenet.m! is used in all experiments.

 We train the models discussed in \cref{s:examples} on ImageNet ILSVRC. \cref{f:speed} reports the training speed as number of images per second processed by stochastic gradient descent. AlexNet trains at about 264 images/s with CuDNN, which is about 40\% faster than the vanilla GPU implementation (using CuBLAS) and more than 10 times faster than using the CPUs. Furthermore, we note that, despite MATLAB overhead, the implementation speed is comparable to Caffe (they report 253 images/s with CuDNN and a Titan -- a slightly slower GPU than the Titan Black used here).  Note also that, as the model grows in size, the size of a SGD batch must be decreased (to fit in the GPU memory), increasing the overhead impact somewhat.

 \cref{f:mgpu} reports the speed on VGG-VD-16, a very large model, using multiple GPUs. In this case, the batch size is set to 264 images. These are further divided in sub-batches of 22 images each to fit in the GPU memory; the latter are then distributed among one to four GPUs on the same machine. While there is a substantial communication overhead, training speed increases from 20 images/s to 45. Addressing this overhead is one of the medium term goals of the library.

\begin{table}
\centering
\begin{tabular}{|lc|ccc|}
  \hline
  model     & batch sz. & CPU  & GPU   & CuDNN \\
  \hline
  AlexNet   & 256       & 22.1 & 192.4 & 264.1 \\
  VGG-F     & 256       & 21.4 & 211.4 & 289.7 \\
  VGG-M     & 128       & 7.8  & 116.5 & 136.6 \\
  VGG-S     & 128       & 7.4  & 96.2  & 110.1 \\
  VGG-VD-16 & 24        & 1.7  & 18.4  & 20.0  \\
  VGG-VD-19 & 24        & 1.5  & 15.7  & 16.5  \\
  \hline
\end{tabular}
\caption{ImageNet training speed (images/s).}
\label{f:speed}
\end{table}

\begin{table}
\centering
\begin{tabular}{|c|cccc|}
  \hline
  num GPUs     & 1  & 2 & 3 & 4 \\
  \hline
  VGG-VD-16 speed & 20.0 & 22.20 & 38.18 & 44.8 \\
  \hline
\end{tabular}
\caption{Multiple GPU speed (images/s).}
\label{f:mgpu}
\end{table}

% ------------------------------------------------------------------
\section{Acknowledgments}\label{s:ack}
% ------------------------------------------------------------------

\matconvnet is a community project, and as such acknowledgements go to all contributors. We kindly thank NVIDIA supporting this project by providing us with top-of-the-line GPUs and MathWorks for ongoing discussion on how to improve the library.

The implementation of several CNN computations in this library are inspired by the Caffe library~\cite{jia13caffe} (however, Caffe is \emph{not} a dependency). Several of the example networks have been trained by Karen Simonyan as part of~\cite{chatfield14return} and~\cite{simonyan15very}.

% ------------------------------------------------------------------
\chapter{Computational blocks}\label{s:blocks}
% ------------------------------------------------------------------

This chapters describes the individual computational blocks supported by \matconvnet. The interface of a CNN computational block !<block>! is designed after the discussion in \cref{s:fundamentals}. The block is implemented as a MATLAB function !y = vl_nn<block>(x,w)! that takes as input MATLAB arrays !x! and !w! representing the input data and parameters and returns an array !y! as output. In general, !x! and !y! are 4D real arrays packing $N$ maps or images, as discussed above, whereas !w! may have an arbitrary shape.

The function implementing each block is capable of working in the backward direction as well, in order to compute derivatives. This is done by passing a third optional argument !dzdy! representing the derivative of the output of the network with respect to $\by$; in this case, the function returns the derivatives ![dzdx,dzdw] = vl_nn<block>(x,w,dzdy)! with respect to the input data and parameters. The arrays !dzdx!, !dzdy! and !dzdw! have the same dimensions of !x!, !y! and !w! respectively (see \cref{s:back}).

Different functions may use a slightly different syntax, as needed: many functions can take additional optional arguments, specified as property-value pairs; some do not have parameters  !w! (e.g. a rectified linear unit); others can take multiple inputs and parameters, in which case there may be more than one !x!, !w!, !dzdx!, !dzdy! or !dzdw!. See the rest of the chapter and MATLAB inline help for details on the syntax.\footnote{Other parts of the library will wrap these functions into objects with a perfectly uniform interface; however, the low-level functions aim at providing a straightforward and obvious interface even if this means differing slightly from block to block.}

The rest of the chapter describes the blocks implemented in \matconvnet, with a particular focus on their analytical definition. Refer instead to MATLAB inline help for further details on the syntax.

% ------------------------------------------------------------------
\section{Convolution}\label{s:convolution}
% ------------------------------------------------------------------

\begin{figure}[t]
	\centering
	\includegraphics[width=0.7\textwidth]{figures/svg/conv}
	\caption{\textbf{Convolution.} The figure illustrates the process of filtering a 1D signal $\bx$ by a filter $f$ to obtain a signal $\by$. The filter has $H'=4$ elements and is applied with a stride of $S_h =2$ samples. The purple areas represented padding $P_-=2$ and $P_+=3$ which is zero-filled. Filters are applied in a sliding-window manner across the input signal. The samples of $\bx$ involved in the calculation of a sample of $\by$ are shown with arrow. Note that the rightmost sample of $\bx$  is never processed by any filter application due to the sampling step. While in this case the sample is in the padded region, this can happen also without padding.}\label{f:conv}
\end{figure}

The convolutional block is implemented by the function !vl_nnconv!. !y=vl_nnconv(x,f,b)! computes the convolution of the input map $\bx$ with a bank of $K$ multi-dimensional filters $\bff$ and biases $b$. Here
\[
 \bx\in\real^{H \times W \times D}, \quad
 \bff\in\real^{H' \times W' \times D \times D''}, \quad
 \by\in\real^{H'' \times W'' \times D''}.
\]
The process of convolving a signal is illustrated in \cref{f:conv} for a 1D slice. Formally, the output is given by
\[
y_{i''j''d''}
=
b_{d''}
+
\sum_{i'=1}^{H'}
\sum_{j'=1}^{W'}
\sum_{d'=1}^D
f_{i'j'd} \times x_{i''+i'-1,j''+j'-1,d',d''}.
\]
The call !vl_nnconv(x,f,[])! does not use the biases. Note that the function works with arbitrarily sized inputs and filters (as opposed to, for example, square images). See \cref{s:impl-convolution} for technical details.

\paragraph{Padding and stride.} !vl_nnconv! allows to specify  top-bottom-left-right paddings $(P_h^-,P_h^+,P_w^-,P_w^+)$ of the input array and subsampling strides $(S_h,S_w)$ of the output array:
\[
y_{i''j''d''}
=
b_{d''}
+
\sum_{i'=1}^{H'}
\sum_{j'=1}^{W'}
\sum_{d'=1}^D
f_{i'j'd} \times x_{S_h (i''-1)+i'-P_h^-, S_w(j''-1)+j' - P_w^-,d',d''}.
\]
In this expression, the array $\bx$ is implicitly extended with zeros as needed.

\paragraph{Output size.} !vl_nnconv! computes only the ``valid'' part of the convolution; i.e. it requires each application of a filter to be fully contained in the input support.  The size of the output is computed in \cref{s:receptive-simple-filters} and is given by:
\[
  H'' = 1 + \left\lfloor \frac{H - H' + P_h^- + P_h^+}{S_h} \right\rfloor.
\]
Note that the padded input must be at least as large as the filters: $H +P_h^- + P_h^+ \geq H'$, otherwise an error is thrown.

\paragraph{Receptive field size and geometric transformations.} Very often it is useful to geometrically relate the indexes of the various array to the input data (usually images) in terms of coordinate transformations and size of the receptive field (i.e. of the image region that affects an output). This is derived in \cref{s:receptive-simple-filters}.

\paragraph{Fully connected layers.} In other libraries, \emph{fully connected blocks or layers} are linear functions where each output dimension depends on all the input dimensions. \matconvnet does not distinguish between fully connected layers and convolutional blocks. Instead, the former is a special case of the latter obtained when the output map $\by$ has dimensions $W''=H''=1$. Internally, !vl_nnconv! handles this case more efficiently when possible.

\paragraph{Filter groups.} For additional flexibility, !vl_nnconv! allows to group channels of the input array $\bx$ and apply different subsets of filters to each group. To use this feature, specify as input a bank  of $D''$ filters $\bff\in\real^{H'\times W'\times D'\times D''}$ such that $D'$ divides the number of input dimensions $D$. These are treated as $g=D/D'$ filter groups; the first group is applied to dimensions $d=1,\dots,D'$ of the input $\bx$; the second group to dimensions $d=D'+1,\dots,2D'$ and so on. Note that the output is still an array $\by\in\real^{H''\times W''\times D''}$.

An application of grouping is implementing the Krizhevsky and Hinton network~\cite{krizhevsky12imagenet} which uses two such streams. Another application is sum pooling; in the latter case, one can specify $D$ groups of $D'=1$ dimensional filters identical filters of value 1 (however, this is considerably slower than calling the dedicated pooling function as given in \cref{s:pooling}).

\paragraph{Dilation.} !vl_nnconv! allows kernels to be spatially dilated on the fly by inserting zeros between elements. For instance, a dilation factor $d=2$ causes the 1D kernel $[f_1,f_2]$ to be implicitly transformed in the kernel $[f_1,0,0,f_2]$. Thus, with dilation factors $d_h,d_w$, a filter of size $(H_f,W_f)$ is equivalent to a filter of size:
\[
  H' = d_h(H_f - 1) + 1,
  \qquad
  W' = d_w(W_f - 1) + 1.
\]
With dilation, the convolution becomes:
\[
y_{i''j''d''}
=
b_{d''}
+
\sum_{i'=1}^{H_f}
\sum_{j'=1}^{W_f}
\sum_{d'=1}^D
f_{i'j'd} \times x_{
S_h (i''-1)+d_h(i'-1)-P_h^-+1,
S_w (j''-1)+d_w(j'-1)-P_w^-+1,
d',d''}.
\]


% ------------------------------------------------------------------
\section{Convolution transpose (deconvolution)}\label{s:convt}
% ------------------------------------------------------------------

\begin{figure}[t]
	\centering
	\includegraphics[width=0.7\textwidth]{figures/svg/convt}
	\caption{\textbf{Convolution transpose.} The figure illustrates the process of filtering a 1D signal $x$ by a filter $f$ to obtain a signal $y$. The filter is applied as a sliding-window, forming a pattern which is the transpose of the one of \cref{f:conv}. The filter has $H'=4$ samples in total, although each filter application uses two of them (blue squares) in a circulant manner. The purple areas represent crops with $C_-=2$ and $C_+=3$ which are discarded. The arrows exemplify which samples of $x$ are involved in the calculation of a particular sample of $y$. Note that, differently from the forward convolution \cref{f:conv}, there is no need to add padding to the input array; instead, the convolution transpose filters can be seen as being applied with maximum input padding (more would result in zero output values), and the latter can be reduced by cropping the output instead.}\label{f:convt}
\end{figure}

The \emph{convolution transpose} block (sometimes referred to as ``deconvolution'') is the transpose of the convolution block described in \cref{s:convolution}. In \matconvnet, convolution transpose is  implemented by the function !vl_nnconvt!.

In order to understand convolution transpose, let:
\[
 \bx\in\real^{H \times W \times D}, \quad
 \bff\in\real^{H' \times W' \times D \times D''}, \quad
 \by\in\real^{H'' \times W'' \times D''}, \quad
\]
be the input tensor, filters, and output tensors. Imagine operating in the reverse direction by using the filter bank $\bff$ to convolve the output $\by$ to obtain the input $\bx$, using the definitions given in~\cref{s:convolution} for the convolution operator; since convolution is linear, it can be expressed as a matrix $M$ such that  $\vv \bx = M \vv\by$; convolution transpose computes instead $\vv \by = M^\top \vv \bx$. This process is illustrated for a 1D slice in \cref{f:convt}.

There are two important applications of convolution transpose. The first one is the so called \emph{deconvolutional networks}~\cite{zeiler14visualizing} and other networks such as convolutional decoders that use the transpose of a convolution. The second one is implementing data interpolation. In fact, as the convolution block supports input padding and output downsampling, the convolution transpose block supports input upsampling and output cropping.

Convolution transpose can be expressed in closed form in the following rather unwieldy expression (derived in \cref{s:impl-convolution-transpose}):
\begin{multline}\label{e:convt}
y_{i''j''d''} =
 \sum_{d'=1}^{D}
 \sum_{i'=0}^{q(H',S_h)}
 \sum_{j'=0}^{q(W',S_w)}
f_{
1+ S_hi' + m(i''+ P_h^-, S_h),\ %
1+ S_wj' + m(j''+ P_w^-, S_w),\ %
d'',
d'
}
\times \\
x_{
1 - i' + q(i''+P_h^-,S_h),\ %
1 - j' + q(j''+P_w^-,S_w),\ %
d'
}
\end{multline}
where
\[
m(k,S) = (k - 1) \bmod S,
\qquad
q(k,n) = \left\lfloor \frac{k-1}{S} \right\rfloor,
\]
$(S_h,S_w)$ are the vertical and horizontal \emph{input upsampling factors},  $(P_h^-,P_h^+,P_h^-,P_h^+)$ the \emph{output crops}, and $\bx$ and $\bff$ are zero-padded as needed in the calculation. Note also that filter $k$ is stored as a slice $\bff_{:,:,k,:}$ of the 4D tensor $\bff$.

The height of the output array $\by$ is given by
\[
  H'' = S_h (H - 1) + H' -P^-_h - P^+_h.
\]
A similar formula holds true for the width. These formulas are derived in \cref{s:receptive-convolution-transpose} along with an expression for the receptive field of the operator.

We now illustrate the action of convolution transpose in an example (see also \cref{f:convt}).  Consider a 1D slice in the vertical direction, assume that the crop parameters are zero, and that $S_h>1$. Consider the output sample $y_{i''}$ where the index $i''$ is chosen such that $S_h$ divides $i''-1$; according to~\eqref{e:convt}, this sample is obtained as a weighted summation of $x_{i'' / S_h},x_{i''/S_h-1},...$ (note that the order is reversed). The weights are the filter elements $f_1$, $f_{S_h}$,$f_{2S_h},\dots$ subsampled with a step of $S_h$. Now consider computing the element $y_{i''+1}$; due to the rounding in the quotient operation $q(i'',S_h)$, this output sample is obtained as a weighted combination of the same elements of the input $x$ that were used to compute $y_{i''}$; however, the filter weights are now shifted by one place to the right: $f_2$, $f_{S_h+1}$,$f_{2S_h+1}$, $\dots$. The same is true for $i''+2, i'' + 3,\dots$ until we hit $i'' + S_h$. Here the cycle restarts after shifting $\bx$ to the right by one place. Effectively, convolution transpose works as an \emph{interpolating filter}.

% ------------------------------------------------------------------
\section{Spatial pooling}\label{s:pooling}
% ------------------------------------------------------------------

!vl_nnpool! implements max and sum pooling. The \emph{max pooling} operator computes the maximum response of each feature channel in a $H' \times W'$ patch
\[
y_{i''j''d} = \max_{1\leq i' \leq H', 1 \leq j' \leq W'} x_{i''+i'-1,j''+j'-1,d}.
\]
resulting in an output of size $\by\in\real^{H''\times W'' \times D}$, similar to the convolution operator of \cref{s:convolution}. Sum-pooling computes the average of the values instead:
\[
y_{i''j''d} = \frac{1}{W'H'}
\sum_{1\leq i' \leq H', 1 \leq j' \leq W'} x_{i''+i'-1,j''+j'-1,d}.
\]
Detailed calculation of the derivatives is provided in \cref{s:impl-pooling}.

\paragraph{Padding and stride.} Similar to the convolution operator of \cref{s:convolution}, !vl_nnpool! supports padding the input; however, the effect is different from padding in the convolutional block as pooling regions straddling the image boundaries are cropped. For max pooling, this is equivalent to extending the input data with $-\infty$; for sum pooling, this is similar to padding with zeros, but the normalization factor at the boundaries is smaller to account for the smaller integration area.

% ------------------------------------------------------------------
\section{Activation functions}\label{s:activation}
% ------------------------------------------------------------------

\matconvnet supports the following activation functions:
%
\begin{itemize}
\item \emph{ReLU.} !vl_nnrelu! computes the \emph{Rectified Linear Unit} (ReLU):
\[
 y_{ijd} = \max\{0, x_{ijd}\}.
\]

\item \emph{Sigmoid.} !vl_nnsigmoid! computes the \emph{sigmoid}:
\[
 y_{ijd} = \sigma(x_{ijd}) = \frac{1}{1+e^{-x_{ijd}}}.
\]
\end{itemize}
%
See \cref{s:impl-activation} for implementation details.

% ------------------------------------------------------------------
\section{Spatial bilinear resampling}\label{s:spatial-sampler}
% ------------------------------------------------------------------

!vl_nnbilinearsampler! uses bilinear interpolation to spatially warp the image according to an input transformation grid. This operator works with an input image $\bx$, a grid $\bg$, and an output image $\by$ as follows:
\[
  \bx \in \mathbb{R}^{H \times W \times C},
  \qquad
  \bg \in [-1,1]^{2 \times H' \times W'},
  \qquad
  \by \in \mathbb{R}^{H' \times W' \times C}.
\]
The same transformation is applied to all the features channels in the input, as follows:
\begin{equation}\label{e:bilinear}
  y_{i''j''c}
  =
  \sum_{i=1}^H
  \sum_{j=1}^W
  x_{ijc}
  \max\{0, 1-|\alpha_v g_{1i''j''} + \beta_v - i|\}
  \max\{0, 1-|\alpha_u g_{2i''j''} + \beta_u - j|\},
\end{equation}
where, for each feature channel $c$, the output $y_{i''j''c}$ at the location $(i'',j'')$, is a weighted sum of the input values $x_{ijc}$ in the neighborhood of location $(g_{1i''j''},g_{2i''j''})$. The weights, as given in \eqref{e:bilinear}, correspond to performing bilinear interpolation. Furthermore, the grid coordinates are expressed not in pixels, but relative to a reference frame that extends from $-1$ to $1$ for all spatial dimensions of the input image; this is given by choosing the coefficients as:
\[
\alpha_v = \frac{H-1}{2},\quad
\beta_v = -\frac{H+1}{2},\quad
\alpha_u = \frac{W-1}{2},\quad
\beta_u = -\frac{W+1}{2}.
\]

See \cref{s:impl-sampler} for implementation details.

% ------------------------------------------------------------------
\section{Region of interest pooling}\label{s:roi-pooling}
% ------------------------------------------------------------------

The \emph{region of interest (ROI) pooling} block applies max or average pooling to specified subwindows of a tensor. A region is a rectangular region $R = (u_-,v_-,u_+,v_+)$. The region itself is partitioned into $(H',W')$ tiles along the vertical and horizontal directions. The edges of the tiles have coordinates
\begin{align*}
   v_{i'} &= v_- + (v_+ - v_- + 1) (i' - 1), \quad i' = 1,\dots,H',\\
   u_{j'} &= u_- + (u_+ - u_- + 1) (j' - 1), \quad j' = 1,\dots,W'.
\end{align*}
Following the implementation of~\cite{girshick15fast}, the $H'\times W'$ pooling tiles are given by
\[
   \Omega_{i'j'} =
   \{\lfloor v_{i'} \rfloor + 1, \dots, \lceil v_{i'+1} \rceil\}
   \times
   \{\lfloor u_{i'} \rfloor + 1, \dots, \lceil u_{i'+1} \rceil\}.
\]
Then the input and output tensors are as follows:
\[
  \bx \in \mathbb{R}^{H \times W \times C},
  \qquad
  \by \in \mathbb{R}^{H' \times W' \times C},
\]
where
\[
   y_{i'j'c} = \operatornamewithlimits{max}_{(i,j) \in \Omega_{i'j'}} x_{ijc}.
\]
Alternatively, $\max$ can be replaced by the averaging operator.

The extent of each region is defined by four coordinates as specified above; however, differently from tensor indexes, these use $(0,0)$ as the coordinate of the top-left pixel. In fact, if there is a single tile ($H'=W'=1$), then the region $(0,0,H-1,W-1)$ covers the whole input image:
\[
   \Omega_{11} =
   \{1, \dots, W\}
   \times
   \{1, \dots, H\}.
\]

In more details, the input of the block is a sequence of $K$ regions. Each region pools one of the $T$ images in the batch stored in $\bx \in \mathbb{R}^{H\times W\times C\times T}$. Regions are therefore specified as a tensor $R \in \mathbb{R}^{5 \times K}$, where the first coordinate is the index of the pooled image in the batch. The output is a $\by \in \mathbb{R}^{H' \times W' \times C \times K}$ tensor.

For compatibility with~\cite{girshick15fast}, furthermore, the region coordinates are rounded to the nearest integer before the definitions above are used. Note also that, due to the discretization details, 1) tiles always contain at least one pixel, 2) there can be a pixel of overlap between them and 3) the discretization has a slight bias towards left-top pixels.

% ------------------------------------------------------------------
\section{Normalization}\label{s:normalization}
% ------------------------------------------------------------------

% ------------------------------------------------------------------
\subsection{Local response normalization (LRN)}\label{s:ccnormalization}
% ------------------------------------------------------------------

!vl_nnnormalize! implements the Local Response Normalization (LRN) operator. This operator is applied independently at each spatial location and to groups of feature channels as follows:
\[
 y_{ijk} = x_{ijk} \left( \kappa + \alpha \sum_{t\in G(k)} x_{ijt}^2 \right)^{-\beta},
\]
where, for each output channel $k$, $G(k) \subset \{1, 2, \dots, D\}$ is a corresponding subset of input channels. Note that input $\bx$ and output $\by$ have the same dimensions. Note also that the operator is applied uniformly at all spatial locations.

See \cref{s:impl-ccnormalization} for implementation details.

% ------------------------------------------------------------------
\subsection{Batch normalization}\label{s:bnorm}
% ------------------------------------------------------------------

!vl_nnbnorm! implements batch normalization~\cite{ioffe2015}. Batch normalization is somewhat different from other neural network blocks in that it performs computation across images/feature maps in a batch (whereas most blocks process different images/feature maps individually). !y = vl_nnbnorm(x, w, b)! normalizes each channel of the feature map $\mathbf{x}$ averaging over spatial locations and batch instances. Let $T$ be the batch size; then
\[
\mathbf{x}, \mathbf{y} \in \mathbb{R}^{H \times W \times K \times T},
\qquad\mathbf{w} \in \mathbb{R}^{K},
\qquad\mathbf{b} \in \mathbb{R}^{K}.
\]
Note that in this case the input and output arrays are explicitly treated as 4D tensors in order to work with a batch of feature maps. The tensors  $\mathbf{w}$ and $\mathbf{b}$ define component-wise multiplicative and additive constants. The output feature map is given by
\[
y_{ijkt} = w_k \frac{x_{ijkt} - \mu_{k}}{\sqrt{\sigma_k^2 + \epsilon}} + b_k,
\quad
\mu_{k} = \frac{1}{HWT}\sum_{i=1}^H \sum_{j=1}^W \sum_{t=1}^{T} x_{ijkt},
\quad
\sigma^2_{k} = \frac{1}{HWT}\sum_{i=1}^H \sum_{j=1}^W \sum_{t=1}^{T} (x_{ijkt} - \mu_{k})^2.
\]
See \cref{s:impl-bnorm} for implementation details.

% ------------------------------------------------------------------
\subsection{Spatial normalization}\label{s:spnorm}
% ------------------------------------------------------------------

!vl_nnspnorm! implements spatial normalization. The spatial normalization operator acts on different feature channels independently and rescales each input feature by the energy of the features in a local neighbourhood . First, the energy of the features in a neighbourhood $W'\times H'$ is evaluated
\[
n_{i''j''d}^2 = \frac{1}{W'H'}
\sum_{1\leq i' \leq H', 1 \leq j' \leq W'} x^2_{
i''+i'-1-\lfloor \frac{H'-1}{2}\rfloor,
j''+j'-1-\lfloor \frac{W'-1}{2}\rfloor,
d}.
\]
In practice, the factor $1/W'H'$ is adjusted at the boundaries to account for the fact that neighbors must be cropped. Then this is used to normalize the input:
\[
y_{i''j''d} = \frac{1}{(1 + \alpha n_{i''j''d}^2)^\beta} x_{i''j''d}.
\]
See \cref{s:impl-spnorm} for implementation details.

% ------------------------------------------------------------------
\subsection{Softmax}\label{s:softmax}
% ------------------------------------------------------------------

!vl_nnsoftmax! computes the softmax operator:
\[
 y_{ijk} = \frac{e^{x_{ijk}}}{\sum_{t=1}^D e^{x_{ijt}}}.
\]
Note that the operator is applied across feature channels and in a convolutional manner at all spatial locations. Softmax can be seen as the combination of an activation function (exponential) and a normalization operator. See \cref{s:impl-softmax} for implementation details.

% ------------------------------------------------------------------
\section{Categorical losses}\label{s:losses}
% ------------------------------------------------------------------

The purpose of a categorical loss function $\ell(\bx,\bc)$ is to compare a prediction $\bx$ to a ground truth class label $\bc$. As in the rest of \matconvnet, the loss is treated as a convolutional operator, in the sense that the loss is evaluated independently at each spatial location. However, the contribution of different samples are summed together (possibly after weighting) and the output of the loss is a scalar. \Cref{s:loss-classification} losses useful for multi-class classification and the \cref{s:loss-attributes} losses useful for binary attribute prediction. Further technical details are in \cref{s:impl-losses}. !vl_nnloss! implements the following all of these.

% ------------------------------------------------------------------
\subsection{Classification losses}\label{s:loss-classification}
% ------------------------------------------------------------------

Classification losses decompose additively as follows:
\begin{equation}\label{e:addloss}
\ell(\bx,\bc) = \sum_{ijn} w_{ij1n} \ell(\bx_{ij:n}, \bc_{ij:n}).
\end{equation}
Here $\bx \in \mathbb{R}^{H \times W \times C \times N}$ and $\bc \in \{1, \dots, C\}^{H \times W \times 1 \times N}$, such that the slice $\bx_{ij:n}$ represent a vector of $C$ class scores and and $c_{ij1n}$ is the ground truth class label. The !`instanceWeights`! option can be used to specify the tensor $\bw$ of weights, which are otherwise set to all ones; $\bw$ has the same dimension as $\bc$.

Unless otherwise noted, we drop the other indices and denote by $\bx$ and $c$  the slice $\bx_{ij:n}$ and the scalar $c_{ij1n}$. !vl_nnloss! automatically skips all samples such that $c=0$, which can be used as an ``ignore'' label.

\paragraph{Classification error.} The classification error is zero if class $c$ is assigned the largest score and zero otherwise:
\begin{equation}\label{e:loss-classerror}
\ell(\bx,c) = \mathbf{1}\left[c \not= \argmax_k x_c\right].
\end{equation}
Ties are broken randomly.

\paragraph{Top-$K$ classification error.} The top-$K$ classification error is zero if class $c$ is within the top $K$ ranked scores:
\begin{equation}\label{e:loss-classerror}
\ell(\bx,c) = \mathbf{1}\left[ |\{k : x_k \geq x_c \}| \leq K \right].
\end{equation}
The classification error is the same as the top-$1$ classification error.

\paragraph{Log loss or negative posterior log-probability.} In this case, $\bx$ is interpreted as a vector of posterior probabilities $p(k) = x_k, k=1,\dots, C$ over the $C$ classes. The loss is the negative log-probability of the ground truth class:
\begin{equation}\label{e:loss-log}
	\ell(\bx, c) = - \log x_c.
\end{equation}
Note that this makes the implicit assumption $\bx \geq 0, \sum_k x_k = 1$. Note also that, unless $x_c > 0$, the loss is undefined. For these reasons, $\bx$ is usually the output of a block such as softmax that can guarantee these conditions. However, the composition of the naive log loss and softmax is numerically unstable. Thus this is implemented as a special case below.

Generally, for such a loss to make sense, the score $x_c$ should be somehow in competition with the other scores $x_k, k\not = c$. If this is not the case, minimizing \eqref{e:loss-log} can trivially be achieved by maxing all $x_k$ large, whereas the intended effect is that $x_c$ should be large compared to the $x_k, k\not=c$. The softmax block makes the score compete through the normalization factor.

\paragraph{Softmax log-loss or multinomial logistic loss.} This loss combines the softmax block and the log-loss block into a single block:
\begin{equation}\label{e:loss-softmaxlog}
	\ell(\bx, c) = - \log \frac{e^{x_c}}{\sum_{k=1}^C e^{x_k}}
	= - x_c + \log \sum_{k=1}^C e^{x_k}.
\end{equation}
Combining the two blocks explicitly is required for numerical stability. Note that, by combining the log-loss with softmax, this loss automatically makes the score compete: $\ell(bx,c) \approx 0$ when $x_c \gg \sum_{k\not= c} x_k$.

This loss is implemented also in the \emph{deprecated} function !vl_softmaxloss!.

\paragraph{Multi-class hinge loss.} The multi-class logistic loss is given by
\begin{equation}\label{e:loss-multiclasshinge}
	\ell(\bx, c) = \max\{0, 1 - x_c \}.
\end{equation}
Note that $\ell(\bx,c) =0 \Leftrightarrow x_c \geq 1$. This, just as for the log-loss above, this loss does not automatically make the score competes. In order to do that, the loss is usually preceded by the block:
\[
 y_c = x_c - \max_{k \not= c} x_k.
\]
Hence $y_c$ represent the \emph{confidence margin} between class $c$ and the other classes $k \not= c$. Just like softmax log-loss combines softmax and loss, the next loss combines margin computation and hinge loss.

\paragraph{Structured multi-class hinge loss.} The structured multi-class logistic loss, also know as Crammer-Singer loss, combines the multi-class hinge loss with a block computing the score margin:
\begin{equation}\label{e:loss-structuredmulticlasshinge}
	\ell(\bx, c) = \max\left\{0, 1 - x_c + \max_{k \not= c} x_k\right\}.
\end{equation}

% ------------------------------------------------------------------
\subsection{Attribute losses}\label{s:loss-attributes}
% ------------------------------------------------------------------

Attribute losses are similar to classification losses, but in this case classes are not mutually exclusive; they are, instead, binary attributes. Attribute losses decompose additively as follows:
\begin{equation}\label{e:addlossattribute}
\ell(\bx,\bc) = \sum_{ijkn} w_{ijkn} \ell(\bx_{ijkn}, \bc_{ijkn}).
\end{equation}
Here $\bx\in \mathbb{R}^{H \times W \times C \times N}$ and $\bc \in \{-1,+1\}^{H \times W \times C \times N}$, such that the scalar $x_{ijkn}$ represent a confidence that attribute $k$ is on and $c_{ij1n}$ is the ground truth attribute label. The !`instanceWeights`! option can be used to specify the tensor $\bw$ of weights, which are otherwise set to all ones; $\bw$ has the same dimension as $\bc$.

 Unless otherwise noted, we drop the other indices and denote by $x$ and $c$  the scalars $x_{ijkn}$ and  $c_{ijkn}$. As before, samples with $c=0$ are skipped.

\paragraph{Binary error.} This loss is zero only if the sign of $x - \tau$ agrees with the ground truth label $c$:
\begin{equation}\label{e:loss-binary}
 \ell(x,c|\tau) = \mathbf{1}[\sign(x-\tau) \not= c].
\end{equation}
Here $\tau$ is a configurable threshold, often set to zero.

\paragraph{Binary log-loss.} This is the same as the multi-class log-loss but for binary attributes. Namely, this time $x_k \in [0,1]$ is interpreted as the probability that attribute $k$ is on:
\begin{align}\label{e:loss-binarylogloss}
\ell(x,c)
&=
\begin{cases}
- \log x, & c = +1, \\
- \log (1 - x), & c = -1, \\
\end{cases}
\\
&=
- \log \left[ c \left(x - \frac{1}{2}\right) + \frac{1}{2} \right].
\end{align}
Similarly to the multi-class log loss, the assumption $x \in [0,1]$ must be enforced by the block computing $x$.

\paragraph{Binary logistic loss.} This is the same as the multi-class logistic loss, but this time $x/2$ represents the confidence that the attribute is on and $-x/2$ that it is off. This is obtained by using the logistic function $\sigma(x)$
\begin{equation}\label{e:loss-binarylogistic}
 \ell(x,c)
 =
 - \log \sigma(cx)
 =
 -\log \frac{1}{1 + e^{-{cx}}}
 =
 -\log \frac{e^{\frac{cx}{2}}}{e^{\frac{cx}{2}} + e^{-\frac{cx}{2}}}.
\end{equation}

\paragraph{Binary hinge loss.} This is the same as the structured multi-class hinge loss but for binary attributes:
\begin{equation}\label{e:loss-hinge}
\ell(x,c)
=
\max\{0, 1 - cx\}.
\end{equation}
There is a relationship between the hinge loss and the structured multi-class hinge loss which is analogous to the relationship between binary logistic loss and multi-class logistic loss. Namely, the hinge loss can be rewritten as:
\[
\ell(x,c) = \max\left\{0, 1 - \frac{cx}{2} + \max_{k\not= c} \frac{kx}{2}\right\}
\]
Hence the hinge loss is the same as the structure multi-class hinge loss for $C=2$ classes, where $x/2$ is the score associated to class $c=1$ and $-x/2$ the score associated to class $c=-1$.

% ------------------------------------------------------------------
\section{Comparisons}\label{s:comparisons}
% ------------------------------------------------------------------

% ------------------------------------------------------------------
\subsection{$p$-distance}\label{s:pdistance}
% ------------------------------------------------------------------

The !vl_nnpdist! function computes the $p$-distance between the vectors in the input data $\bx$ and a target $\bar\bx$:
\[
  y_{ij} = \left(\sum_d |x_{ijd} - \bar x_{ijd}|^p\right)^\frac{1}{p}
\]
Note that this operator is applied convolutionally, i.e. at each spatial location $ij$ one extracts and compares vectors $x_{ij:}$. By specifying the option !'noRoot', true! it is possible to compute a variant omitting the root:
\[
  y_{ij} = \sum_d |x_{ijd} - \bar x_{ijd}|^p, \qquad p > 0.
\]
See \cref{s:impl-pdistance} for implementation details.

%% ------------------------------------------------------------------
%\subsection{Product}\label{s:product}
%% ------------------------------------------------------------------
%
%\[
% y_{ijd} = x^{(1)}_{ijd} x^{(2)}_{ijd}
%\]
%
%\paragraph{Implementation details.}
%\[
% \frac{dz}{dx^{(1)}_{ijd}}
%  =
% \sum_{i''j''d''}
%  \frac{dz}{dy_{i''j''d''}}
%  \frac{dy_{i''j''d''}}{dx^{(1)}_{ijd}}
%  =
%  \frac{dz}{dy_{ijd''}}
%  x^{(2)}_{ijd},
%  \qquad
%  \frac{dz}{dx^{(2)}_{ijd}}
%   =
%  \frac{dz}{dy_{ijd}}
%  x^{(1)}_{ijd}.
%\]
%
%
%% ------------------------------------------------------------------
%\subsection{Split}\label{s:split}
%% ------------------------------------------------------------------
%
%\[
% y_{ijd}^{(1)} = x_{ijd}, \qquad y_{ijd}^{(2)} = x_{ijd}
%\]
%
%\[
% \frac{dz}{dx_{ijd}} =
%\sum_{i''j''d''}
% \frac{dz}{dy_{i''j''d''}^{(1)}}
%  \frac{dy_{i''j''d''}^{(1)}}{dx_{ijd}}
% +
%  \frac{dz}{dy_{i''j''d''}^{(2)}}
%  \frac{dy_{i''j''d''}^{(2)}}{dx_{ijd}}
%\]

% ------------------------------------------------------------------
\chapter{Wrappers and pre-trained models}\label{s:wrappers}
% ------------------------------------------------------------------

It is easy enough to combine the computational blocks of \cref{s:blocks} ``manually''. However, it is usually much more convenient to use them through a \emph{wrapper} that can implement CNN architectures given a model specification. The available wrappers are briefly summarised in \cref{s:wrappers-overview}.

\matconvnet also comes with many pre-trained models for image classification (most of which are trained on the ImageNet ILSVRC challenge), image segmentation, text spotting, and face recognition. These are very simple to use, as illustrated in \cref{s:pretrained}.

% ------------------------------------------------------------------
\section{Wrappers}\label{s:wrappers-overview}
% ------------------------------------------------------------------

\matconvnet provides two wrappers: SimpleNN for basic chains of blocks (\cref{s:simplenn}) and DagNN for blocks organized in more complex direct acyclic graphs (\cref{s:dagnn}).

% ------------------------------------------------------------------
\subsection{SimpleNN}\label{s:simplenn}
% ------------------------------------------------------------------

The SimpleNN wrapper is suitable for networks consisting of linear chains of computational blocks.  It is largely implemented by the \verb!vl_simplenn! function (evaluation of the CNN and of its derivatives), with a few other support functions such as \verb!vl_simplenn_move! (moving the CNN between CPU and GPU) and \verb!vl_simplenn_display! (obtain and/or print information about the CNN).

\verb!vl_simplenn! takes as input a structure \verb!net! representing the CNN as well as input \verb!x! and potentially output derivatives \verb!dzdy!, depending on the mode of operation. Please refer to the inline help of the \verb!vl_simplenn! function for details on the input and output formats. In fact, the implementation of \verb!vl_simplenn! is a good example of how the basic neural net building blocks can be used together and can serve as a basis for more complex implementations.

% ------------------------------------------------------------------
\subsection{DagNN}\label{s:dagnn}
% ------------------------------------------------------------------

The DagNN wrapper is more complex than SimpleNN as it has to support arbitrary graph topologies. Its design is object oriented, with one class implementing each layer type. While this adds complexity, and makes the wrapper slightly slower for tiny CNN architectures (e.g. MNIST), it is in practice much more flexible and easier to extend.

DagNN is implemented by the \verb!dagnn.DagNN! class (under the \verb!dagnn! namespace).

% ------------------------------------------------------------------
\section{Pre-trained models}\label{s:pretrained}
% ------------------------------------------------------------------

\verb!vl_simplenn! is easy to use with pre-trained models (see the homepage to download some). For example, the following code downloads a model pre-trained on the ImageNet data and applies it to one of MATLAB stock images:
\begin{lstlisting}[language=Matlab]
% setup MatConvNet in MATLAB
run matlab/vl_setupnn

% download a pre-trained CNN from the web
urlwrite(...
  'http://www.vlfeat.org/matconvnet/models/imagenet-vgg-f.mat', ...
  'imagenet-vgg-f.mat') ;
net = load('imagenet-vgg-f.mat') ;

% obtain and preprocess an image
im = imread('peppers.png') ;
im_ = single(im) ; % note: 255 range
im_ = imresize(im_, net.meta.normalization.imageSize(1:2)) ;
im_ = im_ - net.meta.normalization.averageImage ;
\end{lstlisting}
Note that the image should be preprocessed before running the network. While preprocessing specifics depend on the model, the pre-trained model contains a \verb!net.meta.normalization! field that describes the type of preprocessing that is expected. Note in particular that this network takes images of a fixed size as input and requires removing the mean; also, image intensities are normalized in the range [0,255].

The next step is running the CNN. This will return a \verb!res! structure with the output of the network layers:
\begin{lstlisting}[language=Matlab]
% run the CNN
res = vl_simplenn(net, im_) ;
\end{lstlisting}

The output of the last layer can be used to classify the image. The class names are contained in the \verb!net! structure for convenience:
\begin{lstlisting}[language=Matlab]
% show the classification result
scores = squeeze(gather(res(end).x)) ;
[bestScore, best] = max(scores) ;
figure(1) ; clf ; imagesc(im) ;
title(sprintf('%s (%d), score %.3f',...
net.meta.classes.description{best}, best, bestScore)) ;
\end{lstlisting}

Note that several extensions are possible. First, images can be cropped rather than rescaled. Second, multiple crops can be fed to the network and results averaged, usually for improved results. Third, the output of the network can be used as generic features for image encoding.

% ------------------------------------------------------------------
\section{Learning models}\label{s:wrappers-learning}
% ------------------------------------------------------------------

As \matconvnet can compute derivatives of the CNN using backpropagation, it is simple to implement learning algorithms with it. A basic implementation of stochastic gradient descent is therefore straightforward. Example code is provided in \verb!examples/cnn_train!. This code is flexible enough to allow training on NMINST, CIFAR, ImageNet, and probably many other datasets. Corresponding examples are provided in the \verb!examples/! directory.

% ------------------------------------------------------------------
\section{Running large scale experiments}
% ------------------------------------------------------------------

For large scale experiments, such as learning a network for ImageNet, a NVIDIA GPU (at least 6GB of memory) and adequate CPU and disk speeds are highly recommended. For example, to train on ImageNet, we suggest the following:
\begin{itemize}
\item Download the ImageNet data~\url{http://www.image-net.org/challenges/LSVRC}. Install it somewhere and link to it from \verb!data/imagenet12!
\item Consider preprocessing the data to convert all images to have a height of 256 pixels. This can be done with the supplied \verb!utils/preprocess-imagenet.sh! script. In this manner, training will not have to resize the images every time. Do not forget to point the training code to the pre-processed data.
\item Consider copying the dataset into a RAM disk (provided that you have enough memory) for faster access. Do not forget to point the training code to this copy.
\item Compile \matconvnet with GPU support. See the homepage for instructions.
\end{itemize}

Once your setup is ready, you should be able to run \verb!examples/cnn_imagenet! (edit the file and change any flag as needed to enable GPU support and image pre-fetching on multiple threads).

If all goes well, you should expect to be able to train with 200-300 images/sec.
% ------------------------------------------------------------------
\chapter{Geometry}\label{s:geometry}
% ------------------------------------------------------------------

This chapter looks at the geometry of the CNN input-output mapping.

% ------------------------------------------------------------------
\section{Preliminaries}\label{s:preliminaries}
% ------------------------------------------------------------------

In this section we are interested in understanding how components in a CNN depend on components in the layers before it, and in particular on components of the input.  Since CNNs can incorporate blocks that perform complex operations, such as for example cropping their inputs based on data-dependent terms (e.g. Fast R-CNN), this information is generally available only at ``run time'' and cannot be uniquely determined given only the structure of the network. Furthermore, blocks can implement complex operations that are difficult to characterise in simple terms. Therefore, the analysis will be necessarily limited in scope.

We consider blocks such as convolutions for which one can deterministically establish dependency chains between network components. We also assume that all the inputs $\bx$ and outputs $\by$ are in the usual form of spatial maps, and therefore indexed as $x_{i,j,d,k}$ where $i,j$ are spatial coordinates.

Consider a layer $\by = f(\bx)$. We are interested in establishing which components of $\bx$ influence which components of $\by$. We also assume that this relation can be expressed in terms of a sliding rectangular window field, called \emph{receptive field}. This means that the output component  $y_{i'', j''}$ depends only on the input components $x_{i,j}$ where $(i,j) \in \Omega(i'', j'') $ (note that feature channels are implicitly coalesced in this discussion). The set $\Omega(i'',j'')$ is a rectangle defined as follows:
\begin{align}\label{e:receptive}
     i &\in \alpha_h (i'' -1) + \beta_h + \left[- \frac{\Delta_h-1}{2}, \frac{\Delta_h-1}{2}\right] \\
     j &\in \alpha_v (j'' -1) + \beta_v + \left[- \frac{\Delta_v-1}{2}, \frac{\Delta_v-1}{2}\right]
\end{align}
where $(\alpha_h,\alpha_v)$ is the \emph{stride}, $(\beta_h,\beta_v)$ the offset, and $(\Delta_h, \Delta_v)$ the \emph{receptive field size}.

% ------------------------------------------------------------------
\section{Simple filters}\label{s:receptive-simple-filters}
% ------------------------------------------------------------------

We now compute the receptive field geometry $(\alpha_h,\alpha_v,\beta_h,\beta_v,\Delta_h,\Delta_v)$ for the most common operators, namely filters. We consider in particular \emph{simple filters} that are characterised by an integer size, stride, and padding.

It suffices to reason in 1D.  Let $H'$ bet the vertical filter dimension, $S_h$ the subampling stride, and $P_h^-$ and $P_h^+$ the amount of zero padding applied to the top and the bottom of the input $\bx$. Here the value $y_{i''}$ depends on the samples:
\begin{align*}
 x_i : i
 &\in
 [1, H'] + S_h (i'' - 1) - P_h^-
=
\left[-\frac{H'-1}{2}, \frac{H'-1}{2}\right] + S_h (i''-1) - P_h^- + \frac{H'+1}{2}.
\end{align*}
Hence
\[
\alpha_h = S_h,
\qquad
\beta _h = \frac{H'+1}{2} - P_h^-,
\qquad
\Delta_h = H'.
\]
A similar relation holds for the horizontal direction.

Note that many blocks (e.g. max pooling, LNR, ReLU, most loss functions etc.) have a filter-like receptive field geometry. For example, ReLU can be considered a $1 \times 1$ filter, such that $H = S_h=1$ and $P_h^-=P_h^+ =0$. Note that in this case $\alpha_h=1$, $\beta_h=1$ and $\Delta_h=1$.

In addition to computing the receptive field geometry, we are often interested in determining the sizes of the arrays $\bx$ and $\by$ throughout the architecture. In the case of filters, and once more reasoning for a 1D slice, we notice that $y_i''$ can be obtained for $i''=1,2,\dots,H''$ where $H''$ is the largest value of $i''$ before the receptive fields falls outside $\bx$ (including padding). If $H$ is the height of the input array $\bx$, we get the condition
\[
   H' + S_h (H'' - 1) - P_h^- \leq H + P_h^+.
\]
Hence
\begin{equation}\label{e:filtered-height}
   H'' = \left\lfloor \frac{H - H' + P_h^- + P_h^+}{S_h} \right\rfloor + 1.	
\end{equation}

% ------------------------------------------------------------------
\subsection{Pooling in Caffe}
% ------------------------------------------------------------------

MatConvNet treats pooling operators like filters, using the rules above. In the library Caffe, this is done slightly differently, creating some incompatibilities. In their case, the pooling window is allowed to shift enough such that the last application always includes the last pixel of the input. If the stride is greater than one, this means that the last application of the pooling window can be partially outside the input boundaries even if padding is ``officially'' zero.

More formally, if $H'$ is the pool size and $H$ the size of the signal, the last application of the pooling window has index $i'' = H''$ such that
\[
  S_h(i''-1) + H' \big|_{i''= H''} \geq H
  \qquad
  \Leftrightarrow
  \qquad
  H'' = \left\lceil 
  \frac{H - H'}{S_h}
  \right\rceil
  + 1.
\]
If there is padding, the same logic applies after padding the input image, such that the output has height:
\[
H'' = \left\lceil 
  \frac{H - H' + P_h^- + P_h^+}{S_h}
  \right\rceil
  + 1.
\]
This is the same formula as for above filters, but with the ceil instead of floor operator. Note that in practice $P_h^- = P_h^+ = P_h$ since Caffe does not support asymmetric padding. 

Unfortunately, it gets more complicated. Using the formula above, it can happen that the last padding application is completely outside the input image and Caffe tries to avoid it. This requires
\begin{equation}\label{e:pooling-caffe-constr}
  S(i'' - 1) - P_h^- + 1 \big|_{i''= H''} \leq H
  \qquad
  \Leftrightarrow
  \qquad
  H'' \leq \frac{H - 1 + P_h^-}{S_h} + 1.	
\end{equation}

Using the fact that for integers $a,b$, one has $\lceil a/b \rceil = \lfloor (a+b-1)/b \rfloor$, we can rewrite the expression for $H''$ as follows
\begin{align*}
H'' = \left\lceil 
  \frac{H - H' + P_h^- + P_h^+}{S_h}
  \right\rceil
  + 1
  =
  \left\lfloor
  \frac{H - 1 +P_h^-}{S_h}
  +
  \frac{P^+_h + S_h - H'}{S_h}
  \right\rfloor
  +1.
 \end{align*}
Hence if $P_h^+ +  S_h \leq H' $ then the second term is less than zero and \eqref{e:pooling-caffe-constr} is satisfied. In practice, Caffe assumes that $P_h^+, P_h^- \leq H' -1$, as otherwise the first filter application falls entirely in the padded region.  Hence, we can upper bound the second term:
\[
\frac{P^+_h + S_h - H'}{S_h}
\leq
\frac{S_h - 1}{S_h}
\leq
1.
\]
We conclude that, for any choices of $P_h^+$ and $S_h$ allowed by Caffe, the formula above may violate constraint \eqref{e:pooling-caffe-constr} by at most one unit. Caffe has a special provision for that and lowers $H''$ by one when needed. Furthermore, we see that if $P_h^+=0$ \emph{and} $S_h \leq H'$ (which is often the case and may be assumed by Caffe), then the equation is also satisfied and Caffe skips the check.

Next, we find MatConvNet equivalents for these parameters. Assume that Caffe applies a symmetric padding $P_h$. Then in MatConvNet $P_h^-=P_h$ to align the top part of the output signal. To match Caffe, the last sample of the last filter application has to be on or to the right of the last Caffe-padded pixel:
\[
\underbrace{
S_h
\left(
\underbrace
{
\left\lfloor
\frac{H - H' + P_h^- + P_h^+}{S_h}  + 1 
\right\rfloor
}_{\text{MatConvNet rightmost pooling index}}
- 1
\right)
+ H'
}_{\text{MatConvNet rightmost pooled input sample}}
\geq
\underbrace{
H + 2P_h^-
}_{\text{Caffe rightmost input sample with padding}}.
\]
Rearranging
\[
\left\lfloor
\frac{H - H' + P_h^- + P_h^+}{S_h}
\right\rfloor
\geq
\frac{H - H' + 2P_h^{-}}{S_h}
\]
Using $\lfloor a/b \rfloor = \lceil (a - b + 1)/b\rceil$ we get the \emph{equivalent} condition:
\[
\left\lceil 
\frac{H - H' + 2P_h^-}{S_h} + \frac{P_h^+ - P_h^- - S_h + 1}{S_h}
\right\rceil
\geq
\frac{H - H' + 2P_h^-}{S_h} 
\]
Removing the ceil operator lower bounds the left-hand side of the equation and produces the \emph{sufficient} condition
\[
 P_h^+ \geq P_h^- + S_h - 1.
\]
As before, this may still be too much padding, causing the last pool window application to be entirely in the rightmost padded area. MatConvNet places the restriction $P_h^+ \leq H' -1$, so that
\[
  P_h^+ = \min\{ P_h^- + S_h - 1 , H' - 1\}.
\]
For example, a pooling region of width $H'=3$ samples with  a stride of $S_h=1$ samples and null Caffe padding $P_h^-=0$, would result in a right MatConvNet padding of $P_h^+ = 1$.

% ------------------------------------------------------------------
\section{Convolution transpose}\label{s:receptive-convolution-transpose}
% ------------------------------------------------------------------

The convolution transpose block is similar to a simple filter, but somewhat more complex. Recall that convolution transpose (\cref{s:impl-convolution-transpose}) is the transpose of the convolution operator, which in turn is a filter. Reasoning for a 1D slice, let $x_i$ be the input to the convolution transpose block and $y_{i''}$ its output. Furthermore let $U_h$, $C_h^-$, $C_h^+$ and $H'$ be the upsampling factor, top and bottom crops, and filter height, respectively.

If we look at the convolution transpose backward, from the output to the input (see also \cref{f:convt}), the data dependencies are the same as for the convolution operator, studied in \cref{s:receptive-simple-filters}. Hence there is an interaction between $x_i$ and $y_{i''}$ only if
\begin{equation}\label{e:convt-bounds}
   1 + U_h(i - 1) - C_h^- \leq i'' \leq H' + U_h(i - 1) - C_h^-
\end{equation}
where cropping becomes padding and upsampling becomes downsampling. Turning this relation around, we find that
\[
 \left\lceil \frac{i'' + C_h^- -H'}{S_h} \right\rceil + 1
 \leq
 i
 \leq
 \left\lfloor \frac{i'' + C_h^- - 1}{S_h} \right\rfloor + 1 .
\]
Note that, due to rounding, it is not possible to express this set tightly in the form outlined above. We can however relax these two relations (hence obtaining a slightly larger receptive field) and conclude that
\[
\alpha_h = \frac{1}{U_h},
\qquad
\beta_h = \frac{2C_h^- - H' + 1}{2 U_h} + 1,
\qquad
\Delta_h = \frac{H' -1}{U_h} + 1.
\]

Next, we want to determine the height $H''$ of the output $\by$ of convolution transpose as a function of the heigh $H$ of the input $\bx$ and the other parameters. Swapping input and output in  \eqref{e:filtered-height} results in the constraint:
\[
H = 1+ \left\lfloor \frac{H'' - H' + C_h^- + C_h^+}{U_h} \right\rfloor.
\]
If $H$ is now given as input, it is not possible to recover $H''$ uniquely from this expression; instead, all the following values are possible
\[
   S_h (H-1) +H' -  C_h^- - C_h^+ \leq H'' < S_h H +H' -  C_h^- - C_h^+.
\]
This is due to the fact that $U_h$ acts as a downsampling factor in the standard convolution direction and some of the samples to the right of the convolution input $\by$ may be ignored by the filter (see also \cref{f:conv} and \cref{f:convt}).

Since the height of $\by$ is then determined up to $S_h$ samples, and since the extra samples would be ignored by the computation and stay zero, we choose the tighter definition and set
\[
H'' =  U_h (H-1) +H' -  C_h^- - C_h^+.
\]

% ------------------------------------------------------------------
\section{Transposing receptive fields}\label{s:receptive-transposing}
% ------------------------------------------------------------------

Suppose we have determined that a later $\by = f(\bx)$ has a receptive field transformation $(\alpha_h,\beta_h,\Delta_h)$ (along one spatial slice). Now suppose we are given a block $\bx = g(\by)$ which is the ``transpose'' of $f$, just like the convolution transpose layer is the transpose of the convolution layer. By this, we mean that, if $y_{i''}$ depends on $x_{i}$ due to $f$, then $x_{i}$ depends on $y_{i''}$ due to $g$.

Note that, by definition of receptive fields, $f$ relates the  inputs and outputs index pairs $(i,i'')$ given by \eqref{e:receptive}, which can be rewritten as
\[
- \frac{\Delta_h-1}{2} \leq  i - \alpha_h (i'' -1) - \beta_h \leq\frac{\Delta_h-1}{2}.
\]
A simple manipulation of this expression results in the equivalent expression:
\[
- \frac{(\Delta_h + \alpha_h - 1)/\alpha_h-1}{2} \leq  i'' - \frac{1}{\alpha_h} (i - 1) - \frac{1 + \alpha_h - \beta_h }{\alpha_h} \leq\frac{(\Delta_h + \alpha_h - 1)/\alpha_h-1}{2\alpha_h}.
\]
Hence, in the reverse direction, this corresponds to a RF transformation
\[
\hat \alpha_h = \frac{1}{\alpha_h},
\qquad
\hat \beta_h = \frac{1 + \alpha_h - \beta_h}{\alpha_h},
\qquad
\hat \Delta_h = \frac{\Delta_h + \alpha_h -1}{\alpha_h}.
\]

\begin{example}
For convolution, we have found the parameters:
\[
\alpha_h = S_h,
\qquad
\beta_h = \frac{H'+1}{2} - P_h^-,
\qquad
\Delta_h = H'.
\]
Using the formulas just found, we can obtain the RF transformation for convolution transpose:
\begin{align*}
\hat \alpha_h &= \frac{1}{\alpha_h} = \frac{1}{S_h},
\\
\hat \beta_h &= \frac{1 + S_h - (H'+1)/2 + P_h^-}{S_h}
= \frac{P_h^- -H'/2 +1/2}{S_h} + 1
= \frac{2P_h^- -H' + 1}{S_h} + 1,
\\
\hat \Delta_h &= \frac{H' + S_h - 1}{S_h} = \frac{H' -1}{S_h} + 1.
\end{align*}
Hence we find again the formulas obtained in \cref{s:receptive-convolution-transpose}.
\end{example}


% ------------------------------------------------------------------
\section{Composing receptive fields}\label{s:receptive-composing}
% ------------------------------------------------------------------

Consider now the composition of two layers $h = g \circ f$ with receptive fields $(\alpha_f, \beta_f, \Delta_f)$ and $(\alpha_g, \beta_g, \Delta_g)$ (once again we consider only a 1D slice in the vertical direction, the horizontal one being the same). The goal is to compute the receptive field of $h$.

To do so, pick a sample $i_g$ in the domain of $g$. The first and last sample $i_f$ in the domain of $f$ to affect $i_g$ are given by:
\[
  i_f = \alpha_f (i_g- 1) + \beta_f \pm \frac{\Delta_f - 1}{2}.
\]
Likewise, the first and last sample $i_g$ to affect a given output sample $i_h$ are given by
\[
  i_g = \alpha_g (i_h- 1) + \beta_g \pm \frac{\Delta_g - 1}{2}.
\]
Substituting one relation into the other, we see that the first and last sample $i_f$ in the domain of $g \circ f$ to affect $i_h$ are:
\begin{align*}\
 i_f &= \alpha_f \left(\alpha_g (i_h- 1) + \beta_g \pm \frac{\Delta_g - 1}{2} - 1\right) + \beta_f \pm \frac{\Delta_f - 1}{2}	
 \\
&= \alpha_f\alpha_g (i_h-1)
 + \alpha_f \beta_g - 1 + \beta_f
 \pm \frac{\alpha_f (\Delta_g - 1) + \Delta_f -1}{2}.
\end{align*}
We conclude that
\[
\alpha_h = \alpha_f \alpha_g,
\qquad
\beta_h =  \alpha_f (\beta_g - 1) + \beta_f,
\qquad
\Delta_h = \alpha_f (\Delta_g - 1) + \Delta_f.
\]

% ------------------------------------------------------------------
\section{Overlaying receptive fields}\label{s:receptive-overlying}
% ------------------------------------------------------------------

Consider now the combination $h(f(\bx_1), g(\bx_2))$ where the domains of $f$ and $g$ are the same. Given the rule above, it is possible to compute how each output sample $i_h$ depends on each input sample $i_f$ through $f$ and on each input sample $i_g$ through $g$. Suppose that this gives receptive fields $(\alpha_{hf}, \beta_{hf}, \Delta_{hf})$ and $(\alpha_{hg}, \beta_{hg}, \Delta_{hg})$ respectively. Now assume that the domain of $f$ and $g$ coincide, i.e. $\bx = \bx_1 = \bx_2$. The goal is to determine the combined receptive field.

This is only possible if, and only if, $\alpha = \alpha_{hg} = \alpha_{hf}$. Only in this case, in fact, it is possible to find a sliding window receptive field that tightly encloses the receptive field due to $g$ and $f$ at all points according to formulas~\eqref{e:receptive}. We say that these two receptive fields are \emph{compatible}. The range of input samples $i = i_f = i_g$ that affect any output sample $i_h$ is then given by
\begin{align*}
	  i_\text{max}&=
  \alpha (i_h- 1) + a, & a = \min
  \left\{\beta_{hf}- \frac{\Delta_{hf} - 1}{2}, \beta_g - \frac{\Delta_{hg} - 1}{2}\right\},
  \\
  	  i_\text{min} &=
  \alpha (i_h- 1) + b, & b = \max
  \left\{\beta_{hf}+ \frac{\Delta_{hf} - 1}{2}, \beta_g + \frac{\Delta_{hg} - 1}{2}\right\}.
\end{align*}
We conclude that the combined receptive field is
\[
\alpha = \alpha_{hg} = \alpha_{hf},
\qquad
\beta = \frac{a+b}{2},
\qquad
\delta = b - a + 1.
\]




% ------------------------------------------------------------------
\chapter{Implementation details}\label{s:impl}
% ------------------------------------------------------------------

This chapter contains calculations and details.

% ------------------------------------------------------------------
\section{Convolution}\label{s:impl-convolution}
% ------------------------------------------------------------------

It is often convenient to express the convolution operation in matrix form. To this end, let $\phi(\bx)$ be the \verb!im2row! operator, extracting all $W' \times H'$ patches from the map $\bx$ and storing them as rows of a $(H''W'') \times (H'W'D)$ matrix. Formally, this operator is given by:
\[
   [\phi(\bx)]_{pq} \underset{(i,j,d)=t(p,q)}{=} x_{ijd}
\]
where the correspondence between indexes $(i,j,d)$ and $(p,q)$ is given by the map $(i,j,d) = t(p,q)$ where:
\[
 i = i''+i'-1, \quad
 j = j''+j'-1, \quad
 p = i'' + H'' (j''-1), \quad
 q = i' + H'(j'-1) + H'W' (d-1).
\]
In practice, this map is slightly modified to account for the padding, stride, and dilation factors. It is also useful to define the ``transposed'' operator \verb!row2im!:
\[
   [\phi^*(M)]_{ijd}
   =
   \sum_{(p,q) \in t^{-1}(i,j,d)}
   M_{pq}.
\]
Note that $\phi$ and $\phi^*$ are linear operators. Both can be expressed by a matrix $H\in\real^{(H''W''H'W'D) \times(HWD)}$ such that
\[
  \vv(\phi(\bx)) = H \vv(\bx), \qquad 
  \vv(\phi^*(M)) = H^\top \vv(M).
\]
Hence we obtain the following expression for the vectorized output (see~\cite{kinghorn96integrals}):
\[
 \vv\by = 
 \vv\left(\phi(\bx) F\right)
 =
 \begin{cases}
 (I \otimes \phi(\bx)) \vv F, & \text{or, equivalently,} \\
 (F^\top \otimes I) \vv \phi(\bx),
 \end{cases}
\]
where $F\in\mathbb{R}^{(H'W'D)\times K}$ is the matrix obtained by reshaping the array $\bff$ and $I$ is an identity matrix of suitable dimensions. This allows obtaining the following formulas for the derivatives:
\[
\frac{dz}{d(\vv F)^\top}
=
\frac{dz}{d(\vv\by)^\top}
(I \otimes \phi(\bx))
= \vv\left[ 
\phi(\bx)^\top 
\frac{dz}{dY}
\right]^\top
\]
where $Y\in\real^{(H''W'')\times K}$ is the matrix obtained by reshaping the array $\by$. Likewise:
\[
\frac{dz}{d(\vv \bx)^\top}
=
\frac{dz}{d(\vv\by)^\top}
(F^\top \otimes I)
\frac{d\vv \phi(\bx)}{d(\vv \bx)^\top}
=
\vv\left[ 
\frac{dz}{dY}
F^\top
\right]^\top
H
\]
In summary, after reshaping these terms we obtain the formulas:
\[
\boxed{
\vv\by = 
 \vv\left(\phi(\bx) F\right),
\qquad
\frac{dz}{dF}
=
\phi(\bx)^\top\frac{d z}{d Y},
\qquad
\frac{d z}{d X}
=
\phi^*\left(
\frac{d z}{d Y}F^\top
\right)
}
\]
where $X\in\real^{(HW)\times D}$ is the matrix obtained by reshaping $\bx$. Notably, these expressions are used to implement the convolutional operator; while this may seem inefficient, it is instead a fast approach when the number of filters is large and it allows leveraging fast BLAS and GPU BLAS implementations.

% ------------------------------------------------------------------
\section{Convolution transpose}\label{s:impl-convolution-transpose}
% ------------------------------------------------------------------

In order to understand the definition of convolution transpose, let $\by$ to be obtained from $\bx$ by the convolution operator as defined in \cref{s:convolution} (including padding and downsampling).  Since this is a linear operation, it can be rewritten as $\vv \by = M \vv\bx$ for a suitable matrix $M$; convolution transpose computes instead $\vv \bx = M^\top \vv \by$.  While this is simple to describe in term of matrices, what happens in term of indexes is tricky. In order to derive a formula for the convolution transpose, start from standard convolution (for a 1D signal):
\[
   y_{i''} = \sum_{i'=1}^{H'} f_{i'} x_{S (i''-1) + i' - P_h^-}, 
   \quad
    1 \leq i'' \leq 1 + \left\lfloor \frac{H - H' + P_h^- + P_h^+}{S} \right\rfloor,
\]
where $S$ is the downsampling factor, $P_h^-$ and $P_h^+$ the padding, $H$ the length of the input signal $\bx$ and $H'$ the length of the filter $\bff$. Due to padding, the index of the input data $\bx$ may exceed the range $[1,H]$; we implicitly assume that the signal is zero padded outside this range.

In order to derive an expression of the convolution transpose,  we make use of the identity $\vv \by^\top (M \vv \bx) = (\vv \by^\top M) \vv\bx = \vv\bx^\top (M^\top \vv\by)$. Expanding this in formulas:
\begin{align*}
\sum_{i''=1}^b y_{i''} 
\sum_{i'=1}^{W'} f_{i'} x_{S (i''-1) + i'  -P_h^-}
&=
\sum_{i''=-\infty}^{+\infty}
\sum_{i'=-\infty}^{+\infty} 
y_{i''}\ f_{i'}\ x_{S (i''-1) + i'  -P_h^-}
\\
&=
\sum_{i''=-\infty}^{+\infty}
\sum_{k=-\infty}^{+\infty} 
y_{i''}\ f_{k-S(i'' -1) + P_h^-}\ x_{k}
\\
&=
\sum_{i''=-\infty}^{+\infty}
\sum_{k=-\infty}^{+\infty} 
y_{i''}%
\ %
f_{%
(k-1+ P_h^-) \bmod S +
S \left(1 -i''  + \left\lfloor \frac{k-1+ P_h^-}{S} \right\rfloor\right)+1
}\ x_{k}
\\
&=
\sum_{k=-\infty}^{+\infty} 
x_{k}
\sum_{q=-\infty}^{+\infty}
y_{\left\lfloor \frac{k-1+ P_h^-}{S} \right\rfloor + 2 - q}
\ %
f_{(k-1+ P_h^-)\bmod S +S(q - 1)+1}.
\end{align*}
Summation ranges have been extended to infinity by assuming that all signals are zero padded as needed. In order to recover such ranges, note that $k \in [1,H]$ (since this is the range of elements of $\bx$ involved in the original convolution). Furthermore, $q\geq 1$ is the minimum value of $q$ for which the filter $\bff$ is non zero; likewise, $q\leq \lfloor (H'-1)/S\rfloor +1$ is a fairly tight upper bound on the maximum value (although, depending on $k$, there could be an element less). Hence
\begin{equation}\label{e:convt-step}
 x_k = 
 \sum_{q=1}^{1 + \lfloor \frac{H'-1}{S} \rfloor}
y_{\left\lfloor \frac{k-1+ P_h^-}{S} \right\rfloor + 2 - q}\ %
f_{(k-1+ P_h^-)\bmod S +S(q - 1)+1},
\qquad k=1,\dots, H.
\end{equation}
Note that the summation extrema in \eqref{e:convt-step} can be refined slightly to account for the finite size of $\by$ and $\bw$:
\begin{multline*}
\max\left\{
1, 
\left\lfloor \frac{k-1 + P_h^-}{S} \right\rfloor + 2 - H''
\right\}
\leq q \\
\leq
1 +\min\left\{
\left\lfloor \frac{H'-1-(k-1+ P_h^-)\bmod S}{S} \right\rfloor, 
\left\lfloor \frac{k-1 + P_h^-}{S} \right\rfloor
\right\}.
\end{multline*}
The size $H''$ of the output of convolution transpose is obtained in \cref{s:receptive-convolution-transpose}.

% ------------------------------------------------------------------
\section{Spatial pooling}\label{s:impl-pooling}
% ------------------------------------------------------------------

Since max pooling simply selects for each output element an input element, the relation can be expressed in matrix form as
$
    \vv\by = S(\bx) \vv \bx
$
for a suitable selector matrix $S(\bx)\in\{0,1\}^{(H''W''D) \times (HWD)}$. The derivatives can be written as:
$
\frac{d z}{d (\vv \bx)^\top}
=
\frac{d z}{d (\vv \by)^\top}
S(\bx),
$
for all but a null set of points, where the operator is not differentiable (this usually does not pose problems in optimization by stochastic gradient). For average pooling, similar relations exists with two differences: $S$ does not depend on the input $\bx$ and it is not binary, in order to account for the normalization factors. In summary, we have the expressions:
\begin{equation}\label{e:max-mat}
\boxed{
\vv\by = S(\bx) \vv \bx,
\qquad
\frac{d z}{d \vv \bx}
=
S(\bx)^\top
\frac{d z}{d \vv \by}.
}
\end{equation}



% ------------------------------------------------------------------
\section{Activation functions}\label{s:impl-activation}
% ------------------------------------------------------------------

% ------------------------------------------------------------------
\subsection{ReLU}\label{s:impl-relu}
% ------------------------------------------------------------------

The ReLU operator can be expressed in matrix notation as
\[
\vv\by = \diag\bfs \vv \bx,
\qquad
\frac{d z}{d \vv \bx}
=
\diag\bfs
\frac{d z}{d \vv \by}
\]
where $\bfs = [\vv \bx > 0] \in\{0,1\}^{HWD}$ is an indicator vector.

% ------------------------------------------------------------------
\subsection{Sigmoid}\label{s:impl-sigmoid}
% ------------------------------------------------------------------

The derivative of the sigmoid function is given by
\begin{align*}
\frac{dz}{dx_{ijd}}
&= 
\frac{dz}{d y_{ijd}} 
\frac{d y_{ijd}}{d x_{ijd}}
=
\frac{dz}{d y_{ijd}} 
\frac{-1}{(1+e^{-x_{ijd}})^2} ( - e^{-x_{ijd}})
\\
&=
\frac{dz}{d y_{ijd}} 
y_{ijd} (1 - y_{ijd}).
\end{align*}
In matrix notation:
\[
\frac{dz}{d\bx} = \frac{dz}{d\by} \odot 
\by \odot 
(\bone\bone^\top - \by).
\]


% ------------------------------------------------------------------
\section{Spatial bilinear resampling}\label{s:impl-sampler}
% ------------------------------------------------------------------

The projected derivative $d\langle \bp, \phi(\bx,\bg)\rangle / d\bx$ of the spatial bilinaer resampler operator with respect to the input image $\bx$ can be found as follows:
\begin{multline}\label{e:bilinear-back-x}
  \frac{\partial}{\partial x_{ijc}}
  \left[
  \sum_{i''j''c''}
  p_{i''k''c''}
  \sum_{i'=1}^H
  \sum_{j'=1}^W 
  x_{i'j'c''}
  \max\{0, 1-|\alpha_v g_{1i''j''} + \beta_v -i'|\}
  \max\{0, 1-|\alpha_u g_{2i''j''} + \beta_u -j'|\}
  \right]
  \\
=
  \sum_{i''j''}
  p_{i''k''c}
  \max\{0, 1-|\alpha_v g_{1i''j''} + \beta_v -i|\}
  \max\{0, 1-|\alpha_u g_{2i''j''} + \beta_u -j|\}.
\end{multline}
Note that the formula is similar to Eq.~\ref{e:bilinear}, with the difference that summation is on $i''$ rather than $i$.

The projected derivative $d\langle \bp, \phi(\bx,\bg)\rangle / d\bg$ with respect to the grid is similar:
\begin{multline}\label{e:bilinear-back-g}
  \frac{\partial}{\partial g_{1i'j'}}
  \left[
  \sum_{i''j''c}
  p_{i''k''c}
  \sum_{i=1}^H
  \sum_{j=1}^W 
  x_{ijc}
  \max\{0, 1-|\alpha_v g_{1i''j''} + \beta_v -i|\}
  \max\{0, 1-|\alpha_u g_{2i''j''} + \beta_u -j|\}
  \right]
  \\
=
  -
  \sum_c
  p_{i'j'c}
  \sum_{i=1}^H
  \sum_{j=1}^W
  \alpha_v x_{ijc}
  \max\{0, 1-|\alpha_v g_{2i'j'} + \beta_v -j|\}
  \sign(\alpha_v g_{1i'j'} + \beta_v -j)
  \mathbf{1}_{\{-1 < \alpha_u g_{2i'j'} + \beta_u < 1\}}.
\end{multline}
A similar expression holds for $\partial g_{2i'j'}$

% ------------------------------------------------------------------
\section{Normalization}\label{s:normalization}
% ------------------------------------------------------------------

% ------------------------------------------------------------------
\subsection{Local response normalization (LRN)}\label{s:impl-ccnormalization}
% ------------------------------------------------------------------

The derivative is easily computed as:
\[
\frac{dz}{d x_{ijd}}
=
\frac{dz}{d y_{ijd}}
L(i,j,d|\bx)^{-\beta}
-2\alpha\beta x_{ijd}
\sum_{k:d\in G(k)}
\frac{dz}{d y_{ijk}}
L(i,j,k|\bx)^{-\beta-1} x_{ijk} 
\]
where
\[
 L(i,j,k|\bx) = \kappa + \alpha \sum_{t\in G(k)} x_{ijt}^2.
\]

% ------------------------------------------------------------------
\subsection{Batch normalization}\label{s:impl-bnorm}
% ------------------------------------------------------------------

The derivative of the network output $z$ with respect to the multipliers $w_k$ and biases $b_k$ is given by
\begin{align*}
	\frac{dz}{dw_k} &= \sum_{i''j''k''t''}
\frac{dz}{d y_{i''j''k''t''}} 
\frac{d y_{i''j''k''t''}}{d w_k}
=
\sum_{i''j''t''}
\frac{dz}{d y_{i''j''kt''}} 
\frac{x_{i''j''kt''} - \mu_{k}}{\sqrt{\sigma_k^2 + \epsilon}},
\\
\frac{dz}{db_k} &= \sum_{i''j''k''t''}
\frac{dz}{d y_{i''j''k''t''}} 
\frac{d y_{i''j''k''t''}}{d w_k}
=
\sum_{i''j''t''}
\frac{dz}{d y_{i''j''kt''}}.
\end{align*}

The derivative of the network output $z$ with respect to the block input $x$ is computed as follows:
\[
\frac{dz}{dx_{ijkt}} = \sum_{i''j''k''t''}
\frac{dz}{d y_{i''j''k''t''}} 
\frac{d y_{i''j''k''t''}}{d x_{ijkt}}.
\]
Since feature channels are processed independently, all terms with $k''\not=k$ are zero. Hence
\[
\frac{dz}{dx_{ijkt}} = \sum_{i''j''t''}
\frac{dz}{d y_{i''j''kt''}} 
\frac{d y_{i''j''kt''}}{d x_{ijkt}},
\]
where
\[
\frac{d y_{i''j''kt''}}{d x_{ijkt}} 
=
w_k
\left(\delta_{i=i'',j=j'',t=t''} - \frac{d \mu_k}{d x_{ijkt}}\right)
\frac{1}{\sqrt{\sigma^2_k + \epsilon}}
-
\frac{w_k}{2}
\left(x_{i''j''kt''} - \mu_k\right)
\left(\sigma_k^2 + \epsilon \right)^{-\frac{3}{2}}
\frac{d \sigma_k^2}{d x_{ijkt}},
\]
the derivatives with respect to the mean and variance are computed as follows:
\begin{align*}
\frac{d \mu_k}{d x_{ijkt}} &= \frac{1}{HWT},
\\
\frac{d \sigma_k^2}{d x_{i'j'kt'}}
&=
\frac{2}{HWT}
\sum_{ijt}
\left(x_{ijkt} - \mu_k \right)
\left(\delta_{i=i',j=j',t=t'} - \frac{1}{HWT} \right)
=
\frac{2}{HWT} \left(x_{i'j'kt'} - \mu_k \right),
\end{align*}
and $\delta_E$ is the indicator function of the event $E$. Hence
\begin{align*}
\frac{dz}{dx_{ijkt}}
&=
\frac{w_k}{\sqrt{\sigma^2_k + \epsilon}}
\left(
\frac{dz}{d y_{ijkt}} 
-
\frac{1}{HWT}\sum_{i''j''kt''}
\frac{dz}{d y_{i''j''kt''}} 
\right)
\\
&-
\frac{w_k}{2(\sigma^2_k + \epsilon)^{\frac{3}{2}}}
\sum_{i''j''kt''}
\frac{dz}{d y_{i''j''kt''}} 
\left(x_{i''j''kt''} - \mu_k\right)
\frac{2}{HWT} \left(x_{ijkt} - \mu_k \right)
\end{align*}
i.e.
\begin{align*}
\frac{dz}{dx_{ijkt}}
&=
\frac{w_k}{\sqrt{\sigma^2_k + \epsilon}}
\left(
\frac{dz}{d y_{ijkt}} 
-
\frac{1}{HWT}\sum_{i''j''kt''}
\frac{dz}{d y_{i''j''kt''}} 
\right)
\\
&-
\frac{w_k}{\sqrt{\sigma^2_k + \epsilon}}
\,
\frac{x_{ijkt} - \mu_k}{\sqrt{\sigma^2_k + \epsilon}}
\,
\frac{1}{HWT}
\sum_{i''j''kt''}
\frac{dz}{d y_{i''j''kt''}} 
\frac{x_{i''j''kt''} - \mu_k}{\sqrt{\sigma^2_k + \epsilon}}.
\end{align*}
We can identify some of these terms with the ones computed as derivatives of bnorm with respect to $w_k$ and $\mu_k$:
\begin{align*}
\frac{dz}{dx_{ijkt}}
&=
\frac{w_k}{\sqrt{\sigma^2_k + \epsilon}}
\left(
\frac{dz}{d y_{ijkt}} 
-
\frac{1}{HWT}
\frac{dz}{d b_k} 
-
\frac{x_{ijkt} - \mu_k}{\sqrt{\sigma^2_k + \epsilon}}
\,
\frac{1}{HWT}
\frac{dz}{dw_k}
\right).
\end{align*}

% ------------------------------------------------------------------
\subsection{Spatial normalization}\label{s:impl-spnorm}
% ------------------------------------------------------------------

The neighbourhood norm $n^2_{i''j''d}$ can be computed by applying average pooling to $x_{ijd}^2$ using \verb!vl_nnpool! with a $W'\times H'$ pooling region, top padding $\lfloor \frac{H'-1}{2}\rfloor$, bottom padding $H'-\lfloor \frac{H-1}{2}\rfloor-1$, and similarly for the horizontal padding.

The derivative of spatial normalization can be obtained as follows:
\begin{align*}
\frac{dz}{dx_{ijd}} 
&= \sum_{i''j''}
\frac{dz}{d y_{i''j''d}} 
\frac{d y_{i''j''d}}{d x_{ijd}}
\\
&=
\sum_{i''j''}
\frac{dz}{d y_{i''j''d}} 
(1 + \alpha n_{i''j''d}^2)^{-\beta}
\frac{dx_{i''j''d}}{d x_{ijd}} 
-\alpha\beta
\frac{dz}{d y_{i''j''d}} 
(1 + \alpha n_{i''j''d}^2)^{-\beta-1}
x_{i''j''d}
\frac{dn_{i''j''d}^2}{d (x^2_{ijd})} 
\frac{dx^2_{ijd}}{d x_{ijd}}
\\
&=
\frac{dz}{d y_{ijd}} 
(1 + \alpha n_{ijd}^2)^{-\beta}
-2\alpha\beta x_{ijd}
\left[
\sum_{i''j''}
\frac{dz}{d y_{i''j''d}} 
(1 + \alpha n_{i''j''d}^2)^{-\beta-1}
x_{i''j''d}
\frac{dn_{i''j''d}^2}{d (x_{ijd}^2)}
\right]
\\
&=
\frac{dz}{d y_{ijd}} 
(1 + \alpha n_{ijd}^2)^{-\beta}
-2\alpha\beta x_{ijd}
\left[
\sum_{i''j''}
\eta_{i''j''d}
\frac{dn_{i''j''d}^2}{d (x_{ijd}^2)}
\right],
\quad
\eta_{i''j''d}=
\frac{dz}{d y_{i''j''d}} 
(1 + \alpha n_{i''j''d}^2)^{-\beta-1}
x_{i''j''d}
\end{align*}
Note that the summation can be computed as the derivative of the
\verb!vl_nnpool! block.

% ------------------------------------------------------------------
\subsection{Softmax}\label{s:impl-softmax}
% ------------------------------------------------------------------

Care must be taken in evaluating the exponential in order to avoid underflow or overflow. The simplest way to do so is to divide the numerator and denominator by the exponential of the maximum value:
\[
 y_{ijk} = \frac{e^{x_{ijk} - \max_d x_{ijd}}}{\sum_{t=1}^D e^{x_{ijt}- \max_d x_{ijd}}}.
\]
The derivative is given by:
\[
\frac{dz}{d x_{ijd}}
=
\sum_{k}
\frac{dz}{d y_{ijk}}
\left(
e^{x_{ijd}} L(\bx)^{-1} \delta_{\{k=d\}}
-
e^{x_{ijd}}
e^{x_{ijk}} L(\bx)^{-2}
\right),
\quad
L(\bx) = \sum_{t=1}^D e^{x_{ijt}}.
\]
Simplifying:
\[
\frac{dz}{d x_{ijd}}
=
y_{ijd} 
\left(
\frac{dz}{d y_{ijd}}
-
\sum_{k=1}^K
\frac{dz}{d y_{ijk}} y_{ijk}
\right).
\]
In matrix form:
\[
  \frac{dz}{dX} = Y \odot \left(\frac{dz}{dY} 
  - \left(\frac{dz}{dY} \odot Y\right) \bone\bone^\top\right)
\]
where $X,Y\in\real^{HW\times D}$ are the matrices obtained by reshaping the arrays
$\bx$ and $\by$. Note that the numerical implementation of this expression is straightforward once the output $Y$ has been computed with the caveats above.

% ------------------------------------------------------------------
\section{Categorical losses}\label{s:impl-losses}
% ------------------------------------------------------------------

This section obtains the projected derivatives of the categorical losses in \cref{s:losses}. Recall that all losses give a scalar output, so the projection tensor $p$ is trivial (a scalar).

% ------------------------------------------------------------------
\subsection{Classification losses}\label{s:impl-loss-classification}
% ------------------------------------------------------------------

\paragraph{Top-$K$ classification error.} The derivative is zero a.e.\

\paragraph{Log-loss.} The projected derivative is:
\[
\frac{\partial p \ell(\bx,c)}{\partial x_k}
=
- p \frac{\partial \log (x_c) }{\partial x_k}
=
- p x_c \delta_{k=c}.
\]

\paragraph{Softmax log-loss.} The projected derivative is given by:
\[
\frac{\partial p \ell(\bx,c)}{\partial x_k}
=
- p \frac{\partial}{\partial x_k}
\left(x_c - \log \sum_{t=1}^C e^{x_t}\right)
=
- p \left(\delta_{k=c} - \frac{e^{x_c}}{\sum_{t=1}^C e^{x_t}} \right).
\]
In brackets, we can recognize the output of the loss itself:
\[
 y = \ell(\bx,c) = \frac{e^{x_c}}{\sum_{t=1}^C e^{x_t}}.
\]
Hence the loss derivatives rewrites:
\[
\frac{\partial p \ell(\bx,c)}{\partial x_k}
=
- p \left(\delta_{k=c} - y \right).
\]

\paragraph{Multi-class hinge loss.} The projected derivative is:
\[
\frac{\partial p \ell(\bx,c)}{\partial x_k}
=
- p\,\mathbf{1}[x_c < 1]\,\delta_{k=c}.
\]

\paragraph{Structured multi-class hinge loss.} The projected derivative is:
\[
\frac{\partial p \ell(\bx,c)}{\partial x_k}
=
- p\,\mathbf{1}[x_c < 1 + \max_{t\not= c} x_t]\,(\delta_{k=c} - \delta_{k=t^*}),
\qquad
t^* = \argmax_{t =1,2,\dots,C} x_t.
\]

% ------------------------------------------------------------------
\subsection{Attribute losses}\label{s:impl-loss-attribute}
% ------------------------------------------------------------------

\paragraph{Binary error.} The derivative of the binary error is 0 a.e.

\paragraph{Binary log-loss.} The projected derivative is:
\[
\frac{\partial p \ell(x,c)}{\partial x}
=
- p \frac{c}{c \left(x - \frac{1}{2}\right) + \frac{1}{2}}.
\]

\paragraph{Binary logistic loss.} The projected derivative is:
\[
\frac{\partial p \ell(x,c)}{\partial x}
=
- p \frac{\partial}{\partial x} \log \frac{1}{1+e^{-cx}}
=
- p \frac{c e^{-cx}}{1 + e^{-cx}}
=
- p \frac{c}{e^{cx} + 1}
=
- pc\, \sigma(-cx).
\]

\paragraph{Binary hinge loss.} The projected derivative is
\[
\frac{\partial p \ell(x,c)}{\partial x}
=
- pc\,\mathbf{1}[cx < 1].
\]

% ------------------------------------------------------------------
\section{Comparisons}\label{s:impl-comparisons}
% ------------------------------------------------------------------

% ------------------------------------------------------------------
\subsection{$p$-distance}\label{s:impl-pdistance}
% ------------------------------------------------------------------

The derivative of the operator without root is given by:
\begin{align*}
\frac{dz}{dx_{ijd}}
&=
\frac{dz}{dy_{ij}}
p |x_{ijd} - \bar x_{ijd}|^{p-1} \operatorname{sign} (x_{ijd} - \bar x_{ijd}).
\end{align*}
The derivative of the operator with root is given by:
\begin{align*}
\frac{dz}{dx_{ijd}}
&=
\frac{dz}{dy_{ij}}
\frac{1}{p}
\left(\sum_{d'} |x_{ijd'} - \bar x_{ijd'}|^p \right)^{\frac{1}{p}-1}
p |x_{ijd} - \bar x_{ijd}|^{p-1} \sign(x_{ijd} - \bar x_{ijd})
\\
&= 
\frac{dz}{dy_{ij}}
\frac{|x_{ijd} - \bar x_{ijd}|^{p-1} \sign(x_{ijd} - \bar x_{ijd})}{y_{ij}^{p-1}}, \\
\frac{dz}{d\bar x_{ijd}} &= -\frac{dz}{dx_{ijd}}.
\end{align*}
The formulas simplify a little for $p=1,2$ which are therefore implemented as special cases.


% ------------------------------------------------------------------
\section{Other implementation details}\label{s:impl-others}
% ------------------------------------------------------------------

% ------------------------------------------------------------------
\subsection{Normal sampler}\label{s:impl-normal}
% ------------------------------------------------------------------

The function \verb!vl::randn()! uses the Ziggurah method~\cite{marsaglia00the-ziggurat} to sample from a Normally-distributed random variable. Let $f(x) = \frac{1}{\sqrt{2\pi}} \exp\left(-\frac{1}{2}x^2\right)$ the standard Normal distribution. The sampler encloses $f(x)$ in a simple shape made of $K-1$ horizontal rectangles and a base composed of a rectangle tapering off in an exponential distribution. These are defined by points $x_1 > x_2 > x_3 > \dots > x_K=0$ such that (for the right half of $f(x)$) the layers of the Ziggurat are given by
\[
\forall k=1,\dots,K-1:
\quad R_k = [f(x_k),f(x_{k+1})] \times [0,x_k].
\]
and such that its basis is given by
\[
R_0 = ([0,f(x_{1})] \times [0,x_1]) \cup 
\{ (x,y) : x \geq x_1,\ y \leq f(x_1) \exp(-x_1(x - x_1)) \}
\]
Note that, since the last point $x_K=0$, (half of) the distribution is enclosed by the Ziggurat, i.e. $\forall x \geq 0:(x,f(x)) \in \cup_{k=0}^K R_k$.

The first point $x_1$ in the sequence determines the area of the Ziggurat base:
\[
A = |R_0| = f(x_1)x_1 + f(x_1)/x_1.
\]
The other points are defined recursively such that the area is the same for all rectangles:
\[
A = |R_k| = (f(x_{k+1}) - f(x_k))x_k
\quad\Rightarrow\quad
x_{k+1} = f^{-1} (A/x_k + f(x_k)).
\]
There are two degrees of freedom: the number of subdivisions $K$ and the point $x_1$. Given $K$, the goal is to choose $x_1$ such that the $K$-th points $x_K=0$ lands on zero, enclosing tightly $f(x)$. The required value of $x_1$ is easily found using bisection and, for $K=256$, is $x_1=3.655420419026953$. Given $x_1$, $A$ and all other points in the sequence can be derived easily using the formulas above.

The Ziggurath can be used to quickly sample from the Normal distribution. In order to do so, one first samples a point $(x,y)$ uniformly at random from the Ziggurat $\cup_{k=0}^K R_k$ and then rejects pairs $(x,y)$ that do not belong to the graph of $f(x)$, i.e.\ $y > f(x)$. Specifically:
\begin{enumerate}
\item Sample a point $(x,y)$ uniformly from the Ziggurat. To do so, sample uniformly at random an index $k \in\{0,1,\dots,K-1\}$ and two scalars $u,v$ in the interval $[0,1)$. Then, for $k\geq 1$, set $x = u x_k$ and $y = v f(x_{k+1}) + (1-v)f(x_k)$ (for $k=0$ see below). Since all regions $R_k$ have the same area and $(x,y)$ are then drawn uniformly form the selected rectangle, this samples a point $(x,y)$ from the Ziggurat uniformly at random.
\item If $y \leq f(x)$, accept $x$ as a sample; otherwise, sample again. Note that, when $x \leq x_{k+1}$, the test $y \leq f(x_{k+1}) < f(x)$ is always successful, and the variable $y$ and test can be skipped in the step above.
\end{enumerate}
Next, we complete the procedure for $k=0$, when $R_0$ is not just a rectangle but rather the union of a rectangle and an exponential distribution. To sample from $R_0$ uniformly, we either choose the rectangle or the exponential distribution with a probability proportional to their area. Reusing the notation (and corresponding code) above, we can express this as sampling $x = u x_0$ and accepting the latter as a sample from the rectangle component if $ux_0 \leq x_1$; here the pseudo-point $x_0$ is defined such that $x_1 / x_0 = f(x_1)x_1 / A$, i.e.\ $x_0 = A/f(x_1)$. If the test fails, we sample instead from the exponential distribution $x\sim x_1\exp(-x_1(x-x_1)),$ $x\geq x_1$. To do so, let $z= x_1\exp(-x_1(x-x_1))$; then $x = x_1 - (1/x_1) \ln z/x_1$ and $dx = |- (x_1/z)|dz$, where $z\in(0,x_1]$. Since $x_1\exp(-x_1(x-x_1)) dx = (1/x_1) dz$ is uniform, we can implement this by sampling $u$ uniformly in $(0,1]$ and setting $x = x_1 - (1/x_1) \ln u$. Finally, recall that the goal is to sample from the Normal distribution, not the exponential, so the latter sample must be refined by rejection sampling. As before, this requires sampling a pair $(x,y)$ under the exponential distribution graph. Given $x$ sampled from the exponential distribution, we sample the corresponding $y$ uniformly at random in the interval $[0, f(x_1) \exp(-x_1(x-x_1))]$, and write the latter as $y = v f(x_1) \exp(-x_1(x-x_1))$, where $v$ is uniform in $[0,1]$. The latter is then accepted provided that $y$ is below the Normal distribution graph $f(x)$, i.e. $v f(x_1) \exp(-x_1(x-x_1)) \leq f(x).$ A short calculation yields the test:
\[
-2\ln v \geq x_1^2 +  x^2 - 2x_1x = (x_1 - x)^2 =
((1/x_1) \ln u)^2.
\]

% ------------------------------------------------------------------
\subsection{Euclid's algorithm}\label{s:impl-euclid}
% ------------------------------------------------------------------

Euclid's algorithm finds the \emph{greatest common divisor} (GCD) of two non-negative integers $a$ and $b$. Recall that the GCD is the largest integer that divides both $a$ and $b$:
\[
    \gcd(a,b) = \max\{ d \in \mathbb{N} : d|a \ \wedge\ d|b \}.
\]

\begin{lemma}[Euclid's algorithm]
Let $a,b\in\mathbb{N}$ and let $q\in\mathbb{Z}$ such that $a - qb \geq 0$. Then
$$
\gcd(a,b) = \gcd(a-qb,b).
$$
\end{lemma}
\begin{proof}
Let $d$ be a divisor of both $a$ and $b$. Then $d$ divides $a - qb$ as well because:
$$
\frac{a - qb}{d} = 
\underbrace{\frac{a}{d}}_{\in\mathcal{Z}} - q 
\underbrace{\frac{b}{d}}_{\in\mathbb{Z}}
\quad\Rightarrow\quad
\frac{a - qb}{d} \in \mathbb{Z}.
$$
Hence  $\gcd(a,b) \leq \gcd(q-qb,b)$. In the same way, we can show that, if $d$ divides $a - qb$ as well as $b$, then it must divide $a$ too, hence $\gcd(a-qb, b) \leq\gcd(a,b)$.
\end{proof}

Euclid's algorithm starts with $a > b \geq 1$ and sets $q$ to the quotient of the integer division $a/b$. Due to the lemma above, the GCD of $a$ and $b$ is the same as the GCD of the remainder $r = a - qb = (a \mod b)$ and $b$:
\[
   \gcd(a,b) = \gcd(a, a\mod b).
\]
Since the remainder $(a\mod b) < b$ is strictly smaller than $b$, now GCD is called with smaller arguments. The recursion terminates when a zero reminder is generated, because
\[
   \gcd(a,0) = a.
\]

We can modify the algorithms to also find two integers  $u,v$, the B\'ezout's coefficients, such that:
$$
   a u + bv = \gcd(a,b).
$$
To do so, we replace $a = b (a/b) + r$ as above:
$$
  ru  +  b v'= \gcd(a,b) = \gcd(r,b), \qquad v' = \frac{a}{b} u + v.
$$
The recursion terminates when $r=0$, in which case
$$
  b v'= \gcd(0,b) = b \quad\Rightarrow\quad v'=b.
$$






\clearpage

% ------------------------------------------------------------------
\bibliographystyle{plain}
\bibliography{references,/Users/vedaldi/src/bibliography/vedaldi}
\end{document}
% ------------------------------------------------------------------



